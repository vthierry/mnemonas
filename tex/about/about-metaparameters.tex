\documentclass{article}\usepackage{hyperref}\newcommand{\deq}{\stackrel {\rm def}{=}} \newcommand{\eqline}[1]{\\\centerline{$#1$}\\}\newcommand{\tab}{\hphantom{6mm}}\newcommand{\well}[1]{\vspace{0.3cm}\hspace{-2cm}{\tt #1}} \begin{document}

\section*{Introduction}

\subsection*{Problem position}

Automated machine learning\footnote{The notions of ``\href{https://en.wikipedia.org/wiki/Automated\_machine\_learning}{automated-machine-learning}'', ``\href{https://en.wikipedia.org/wiki/Meta_learning_(computer_science)}{meta-learning}'', including ``\href{https://en.wikipedia.org/wiki/Meta-optimization}{meta-optimization}'' and ``\href{https://en.wikipedia.org/wiki/Hyperparameter_(machine_learning)}{hyperparameter}'' and the related ``\href{http://neupy.com/2016/12/17/hyperparameter\_optimization\_for\_neural\_networks.html}{hyperparameter optimization}'', have precise meaning in machine-learning, and we assume this is known to the reader.} is ``the process of automating the end-to-end process of machine learning'', because the complexity of adjusting the hyperparameters, including selecting a suitable method, becomes easily time-consuming when not intractable. To this end, the general idea is to consider a standard machine-learning algorithm and to add on ``top of it'' another algorithmic mechanism, e.g., another machine learning algorithm dedicated to automatic hyperparameter adjustment, with the caveat of generating other hyperparameters for the meta-learning algorithm and without formal guaranty that this accumulation of mechanisms is optimal.

The key setting is to adjust the model parameter by optimizing the criterion on a {\em learning} data set, and validate the hyperparameter choice by rerunning the optimization for different hyperparameter values on a different {\em validation} data set, while the final performance is evaluated on another {\em test} set. 
Hyperparameter tuning is mainly required to maximize predictive accuracy, i.e., not only optimize the learning on the present data set, but also generalize this learning to subsequent data. Another goal is to not only adjust one model parameter set, but also to adapt the model itself.

Here we propose to follow another track:  How is it possible to consider the automatic adjustment of hyperparameters not as a separate process but within the learning algorithm itself ? Let us call such parameters {\em meta-parameters} and let us attempt to formalize such an alternative track.

\subsection*{Related work}

As being an everyday concrete problem, there is a huge literature on hyperparameter adjustment of optimization algorithms. A recent review of automatic selection methods for machine learning algorithms and hyperparameter values is available \cite{luo2016review}, while the state of the art regarding hyper-heuristics has been made by \cite{burke2013hyper}, an introducing overview about automatic hyperparameter optimization and model selection for supervised machine learning is available in this student work \cite{bermudez2014automatic}. Hyperparameter adjustment packages are available with every machine learning framework, searching through the joint space of hyperparameter settings such as, e.g., Hyperopt \cite{bergstra2013hyperopt} or, e.g., Auto-WEKA \cite{kotthoff2017auto}. They are mainly based on Bayesian optimization method (beyond grid-search or random-search), considering that the cost of one optimization is high, while searching in the hyper-parameter space given a small data set of previous knowledge regarding hyper-parameter performance can highly improve the final result. Such search is based on a surrogate model of the optimization algorithm, using, e.g., tree-structured Parzen estimator \cite{bergstra2011algorithms} or Gaussian process \cite{snoek2012practical} (see \cite{qin2017improving} for a discussion and an important improvement).

Beyond these vanilla methods, several non-trivial improvements have been proposed. Hyperparameters can be updated before model parameters have fully converged (e.g., \cite{pedregosa2016hyperparameter}) interpolating the final optimization criterion value. Train-test samples can be matched across candidate hyperparameter configurations, allowing early elimination of sub-optimal candidates to minimize the number of evaluations and avoiding full cross-validation, with hypothesis testing embedding in the search algorithm \cite{zheng2013lazy}. Cross-validation based protocols with simultaneous hyperparameter optimization is proposed by several authors, e.g., \cite{tsamardinos2015performance}. The key idea of such collaborative hyper-parameter tuning methods \cite{bardenet2013collaborative}, is to design interaction between several hyperparameter evaluation \cite{jamieson2016non}, yielding an unbounded armed bandits problem \cite{li2016efficient}. Among sophisticated hyperparameter adjustment methods exact gradients of cross-validation performance with respect to hyperparameters (chaining derivatives backwards through the training procedure) has been proposed \cite{maclaurin2015gradient}, while hyper-optimization that transfers information by constructing a common response surface for all data set has been developed \cite{yogatama2014efficient}. This tiny review is only illustrative and far from being exhaustive: the topic is really huge (consider, for instance, that this also raise problems of data privacy in the sense of not inferring private data from repeated optimization output \cite{kusner2015differentially}).

In deep link with this issue, authors also consider adaptively learning the model architecture itself \cite{cortes2016adanet}. Automatic model design can be specified as searching for an optimal sub-graph within a large computational graph \cite{pham2018efficient}, these authors using parameter sharing between alternatives to improve the performances. More generally multi-model estimation methods (e.g., \cite{vieville:inria-00000172}) introduce qualitative hyperparameters to adjust.

Finally, another track is to consider algorithms for which hyper-parameter tuning is requires little tuning (e.g., considering the ADAM approach of \cite{kingma2014adam}, where the first-order gradient-based optimization of stochastic objective functions is based on adaptive estimates of lower-order momenta, or improving vanilla or natural gradient methods by automatic adjustment of the local optimization \cite{marceau2016practical}). This includes changing the algorithm itself (e.g., replacing k-mean clustering algorithms, by hierarchical clustering methods).

\subsection*{Proposed contribution}

Let us explore here a rather disruptive track, with respect to these complex issues: What about ``hyperparameterless'' machine learning methods ? In order to draft such approach we are first going to make the distinction between several kind of hyperparameters and claim that the solution depends on such distinction. We then are going to point out that the lever is to introduce as much as possible a-priori information in the machine learning process, indeed, but not at the level decreed by a given method, whereas at the level of the learning design. We further are going to get inspired by the way the brain adjust its ``meta-parameters'' and propose to consider machine learning not only as parameter optimization with meta-learning on a validation set, but as more fexible multi-loop architecture. 

In order to address such issue, we focus on a precise hybrid learning task (mixing supervised and reinforcement learning), with the idea of considering not-so-big data set as discussed in \cite{Drumond2017From}, and input-output interaction. 

and will have to revisit standard learning algorithms.

\section*{Problem setting}

\subsection*{Notations}

The goal is to adjust a parametric distributed input-output computation:
\eqline{{\bf o} = {\bf f}_{\bf w}({\bf i}),}
mapping an input ${\bf i} \in {\cal R}^I$ onto an output ${\bf o} \in {\cal M} \subset {\cal R}^O$, as a function of parameters (e.g., network weights) ${\bf w} \in {\cal R}^W$.

To this end we consider having a temporal sequence of input ${\bf i}_t, t \in \{1, T\}$, coupled with a next-step loss $l_t$, as function of ${\bf f}_{{\bf w}_t}({\bf i}_t)$, the $l_t$ value being available at time $t+1$. 





% No free-lunch theorem for optimization \cite{wolpert1997no}: A general-purpose, universal optimization strategy is impossible. The only way one strategy can outperform another is if it is specialized to the structure of the specific problem under consideration.

\bibliographystyle{alpha}\bibliography{../etc/main.bib} \end{document}
