\documentclass[a4,12pt,twoside]{article}
\newif\ifRR
\RRtrue % \RRfalse or \RRtrue

\ifRR
\usepackage[a4paper]{geometry}
\usepackage{RR}
\graphicspath{{../etc/rr-sty/}}
\else
\usepackage[top=1.5cm,right=4cm,bottom=2cm,left=1.5cm]{geometry}
\fi
\usepackage[T1]{fontenc} \usepackage[utf8]{inputenc}
\usepackage{authblk}
\usepackage{graphicx} 
\usepackage{hyperref}
\usepackage{amsmath}\usepackage{amsfonts}
\newcommand{\deq}{\stackrel {\rm def}{=}}
\newcommand{\eqline}[1]{\\\centerline{$#1$}\\} 
\newcommand{\hr}{\\---------------------------------------------------}
\newcommand{\tab}{\\\hspace{5mm}} 

\usepackage{color}
\definecolor{fred}{RGB}{32,64,128}\newcommand{\fred}[1]{{\color{fred}{{\tt @fred:} #1}}}
\definecolor{thalita}{RGB}{51, 153, 255}\newcommand{\thalita}[1]{{\color{thalita}{{\tt @thalita:} #1}}}
\definecolor{vthierry}{RGB}{80,0,120}\newcommand{\vthierry}[1]{{\color{vthierry}{{\tt @vthierry:} #1}}}
\definecolor{xavier}{RGB}{42,120,42}\newcommand{\xavier}[1]{{\color{xavier}{{\tt @xavier:} #1}}}

\ifRR
\RRNo{9100}
\RRdate{October 2017}
\RRauthor{\thanks[affil]{Mnemosyne team, INRIA Bordeaux}
Thierry Vi\'eville \thanksref{affil} \and
Xavier Hinaut \thanksref{affil} \and
Thalita F. Drumond \thanksref{affil} \and
Fr\'ed\'eric Alexandre \thanksref{affil}}
\authorhead{Alexandre \& Drumond \& Hinault \& Vi\'eville}
\RRtitle{Estimation des poids d'un réseau récurrent \\ par ajustement rétroactif}
\RRetitle{Recurrent neural network weight estimation \\ through backward tuning}
\titlehead{Backward tuning}
\RRresume{Nous considérons une formulation alternative de l'estimation du poids dans les réseaux récurrents, proposant une notation integrant une grande quantité d'unités de réseau récurrentes qui aide à formuler ce problème d'estimation. Réutilisant un «bon vieux» principe de la théorie du contrôle, amélioré ici à l'aide d'une heuristique de stabilisation numérique rétroactive, nous obtenons une estimation distribuée du 2ème ordre, numériquement stable et plutôt efficace, sans aucun méta-paramètre à ajuster. La relation avec les techniques existantes est discutée à chaque étape. La méthode proposée est validée en utilisant des tâches d'ingénierie inverse.}
\RRabstract{We consider another formulation of weight estimation in recurrent networks, proposing a notation for a large amount of recurrent network units that helps formulating the estimation problem. Reusing a ``good old'' control-theory principle, improved here using a continuation heuristic, we obtain a numerically stable and rather efficient second-order and distributed estimation, without any meta-parameter to adjust. The relation with existing technique is discussed at each step. 
The proposed method is validated using reverse engineering tasks and non-trivial numerical computations.

}
\RRmotcle{réseaux récurrents, apprentissage automatique, ajustement rétroactif}
\RRkeyword{recurrent network, machine learning, backward tuning}
\RRprojet{Mnemosyne}
\RCBordeaux
\fi

\begin{document}
\ifRR
\makeRR 
\else
\title{Recurrent neural network weight estimation \\ though backward tuning}
\author[1]{Fr\'ed\'eric Alexandre}
\author[1]{Thalita F. Drumond}
\author[1]{Xavier Hinaut}
\author[1]{Thierry Vi\'eville}
\affil[1]{Mnemosyne team, INRIA Bordeaux}
\maketitle
\begin{abstract}We consider another formulation of weight estimation in recurrent networks, proposing a notation for a large amount of recurrent network units that helps formulating the estimation problem. Reusing a ``good old'' control-theory principle, improved here using a continuation heuristic, we obtain a numerically stable and rather efficient second-order and distributed estimation, without any meta-parameter to adjust. The relation with existing technique is discussed at each step. 
The proposed method is validated using reverse engineering tasks and non-trivial numerical computations.

\end{abstract}
\fi
\iffalse
\fi
\section*{Introduction}

Artificial neural networks can be considered as discrete-time
dynamical systems, performing input-output computation, at the higher
level of generality \cite{siegelmann_turing_1991}. However, only
specific feed-forward or recurrent architectures are considered in
practice, because of weight estimation, as reviewed now.

In the artificial neural network literature, feed-forward networks
parameter learning is a well-solved problem.  For instance, the back-propagation
algorithms, based on specific architectures of multi-layer
feed-forward networks, allows one to propose well-defined
implementation \cite{amit:89}, though it has been shown at the
theoretical and empirical levels, that "shallow" architectures are
 inefficient for representing complex functions
\cite{bengio-lecun:07}.

Deep-networks are specific feed-forward architectures
\cite{bengio-lecun:07} which can have very impressive performances,
e.g.  \cite{farabet_learning_2013}. The key idea
\cite{hstad_power_1991} is that, at least for threshold units with
positive weights, reducing the number of layers induces an exponential
complexity increase for the same input/output function. On the
reverse, it is a reasonable assumption, numerically verified, that
increasing the number of layers yields a compact representation of
input/output functions. One drawback is related to weights supervised
learning in deeper layers, since readout layers may over-fit the
learning set, the remedy being to apply unsupervised learning on
deeper layers (see \cite{Bengio:2009} for an introduction). This
problem desapears with specific architectures such as CNN.

It also remains restrictive by the fact that the architecture is
mainly a pipe-line, including some parallel tracks, while each layer
is a feed-forward network (e.g. a convolutional neural layers) or with
a very specific recurrent connectivity (e.g., restrained Boltzman
machines). In the brain, more general architectures occur (e.g.  with
shortcuts between deeper and lower layers, as it happens in the brain
regarding the pulvinar \cite{citeulike:1590763}) and each layer is a
more general recurrent network (e.g., with short and long range
horizontal connections). Breaking this pipe-line architecture may
overcome the problem of deeper layer weight adjustment.

Feed-forward networks are obviously far from the computational
capacity of recurrent networks. Therefore, specific multi-layer
architectures without recurrent links within a layer and specific
forward/backward connections between layers have been proposed
instead. The first dynamic neural model, the model by Hopfield
\cite{hopfield:82}, appeared much later, and was very specific.
Further solutions include Jordan's network \cite{jordan:86}, Elman's
Networks \cite{elman:90}, Long short term memory (LSTM) by Hochreiter
and Schmidhuber \cite{hochreiter-schmidhuber:97}.

Another track is to consider recurrent networks without supervised
weight adjustment \cite{verstraeten-etal:07}. Units in such
architectures are linear or sigmoid artificial neurons, including
soft-max units, or even spiking neurons. Such network architectures,
such as Echo State Networks \cite{jaeger:03} and Liquid Sate Machines
\cite{maass-etal:02}, are called ``reservoir computing'' (see
\cite{verstraeten-etal:07} for unification of reservoir computing
methods at the experimental level).

In such architectures the recurrent parameters of hidden units is not
explicitly learned, whereas recurrent weights are either randomly
fixed, likely using a sparse connectivity, or adjusted using
unsupervised learning mechanism, without any direct connection with
the learning samples (though the hidden unit statistics, for instance,
is sometimes adjusted in relation with the desired output). In the
case of temporal mechanisms, i.e. using spiking neurons (e.g. in the
model of \cite{paugam-moisy-etal:08}), the unsupervised learning
mechanism of the recurrent weights is a form of synaptic plasticity,
usually STDP (Spike-Time-Dependent Plasticity), a temporal Hebbian
unsupervised learning rule, biologically inspired. It appears that
simple methods yield good results \cite{verstraeten-etal:07}, but
without over-passing recent deep-layer architecture performances
\cite{Deng:2014}.

The general problem of learning recurrent neural networks has also
been widely addressed as reviewed in \cite{cun_theoretical_1988} for
90's studies and in \cite{martens_learning_2016} for recent advances,
and methods exist far beyond basic methods such as backpropagation
through time. 

In the present paper, we revisit the general problem of recurrent
network weight learning, not as it, but because it is
related to modern issues related to both artificial networks and brain
function modeling. Such issues include: Could we adjust the recurrent
weights in a reservoir computing architecture ?  Is it possible to
consider deep-learning architecture, with more general inter and intra
layers connectivity ? Would it be possible to not only use some
specific recurrent architecture as exemplified here, but to learn also
the architecture itself (i.e. learn the weight value and learn if the
connection weight has to be set to zero, cutting the connection) ?

We are not going to address more than weight adjustment in this paper.
For instance learning issues (e.g., boosting
\cite{Freund2003Efficient}) are not within the scope of this paper:
Neither representation learning \cite{Bengio2012Representation}, nor other
complex issues \cite{Goodfellow2016Deep} are considered, this
contribution being only an alternate tool for variational weight
optimization. See \cite{Fdrumond2017} for a recent discussion on 
such issues.

We are also not going to consider biological plausibility in the sense
of \cite{Bengio2016Towards}, but will show that the proposed mehod is
compliant with several distributed biological constraints or
computational properties: local weight adjustment, backward error
propagation, Hebbian like adjustment rules. A more rigourous
discussion about the link with computational neuroscience aspects is
however beyond the scope of this work.

In the next section we choose a notation to state the estimation
problem, and Appendix~\ref{generality} make explicit how this notation
applies to most of the usual frameworks. We then adress the estimation
problem and introduce the modified solution we propose, while
Appendix~\ref{application} further discuss how it can be used for
several estimation problems. In the subsequent section the method is
implemented and numerically evaluated, while
Appendix~\ref{closedforms} expalined why certain estimation problem ar
not considered here because reducing to trivial computation problems,
given an architecture.

\section*{Problem position}

\subsection*{A general recurrent architecture.}

\begin{figure}[!ht]
  \includegraphics[width=0.8\textwidth]{img/recurrent-network}
  \caption{A General recurrent architecture maps a vectorial input sequence ${\bf i}(t)$ onto an output ${\bf o}(t)$, 
    via an internal state ${\bf x}(t)$ of hidden units. It is parameterized by a recurrent weight matrix ${\bf W}$. 
    The dynamics is defined by the network recurrent equations.}
  \label{recurrent-network}
\end{figure}

As schematized in figure~\ref{recurrent-network}, we consider a recurrent network with nodes of the form:
\begin{equation}\label{eq-recurrent}
\begin{array}{rcl} 
x_n(t) &=&  \Phi_{n0t}\left(\cdots, x_{n'}(t'), \cdots, i_{m}(s), \cdots\right) \\ 
          &+& \sum_{d = 1}^{D_{n}} W_{nd} \, \Phi_{ndt}\left(\cdots, x_{n'}(t'), \cdots, i_{m}(s), \cdots\right) \\
o_n(t) &=& x_n(t), n < N_0 \\
 \end{array}
\end{equation} with:
 \\  -  $N$ nodes of value $x_{n}(t)$ indexed by $n \in \{0, N\{$, 
 \\  \hphantom{0.5cm} -  with a maximal state recurrent causal range of $R$ and with either,  
 \\  \hphantom{0.8cm} -  $t - R \leq t' < t$ (i.e., taking into account previous value up to $R$ time-steps in the past) or
 \\  \hphantom{0.8cm} -  $t' = t$ and $n < n'$ (i.e., taking into account present value, of subsequent nodes, in a causal way).
 \\  \hphantom{0.5cm} - Here $N_0 \leq N$ of these nodes are output;
 \\  - $M$ input $i_{m}(s)$ indexed by $m \in \{0, M\{$, $t - S \leq  s < t$, 
 \\  - $1 + D_{n}$ predefined kernels $\Phi_{ndt}\left(\right)$ for each node, defining the network structure;
 \\  - $\sum_n D_n$ static adjustable weights $W_{nd}$, defining the network parameter.

\vphantom{1cm}

Considering equation~(\ref{eq-recurrent}) we notice that : \begin{itemize}

\item The distinction between output or hidden node is simply based on the fact that we can (or not) observe the $o_n(t)$ node value. By convention and without loss of generality, output nodes are the $N_0 \leq N$ first ones.

\item Though, in order to keep compact notations, we mixed node with either 
 \\ \hphantom{0.2cm} - {\em unit firmware} parameter-less function, i.e. with $\Phi_{n0t}()$, or
 \\ \hphantom{0.2cm} - {\em unit learnware} linear combination of elementary kernels, i.e. with $\sum_{d} W_{nd} \, \Phi_{ndt}()$,
\\ in all examples these two kinds of node will be separated. This constraint is not mandatory, but will help clarifying the role of each node.

\item A given state value depends either on previous time values ($t - R \leq t' < t$) or subsequent indexed nodes ($t' = t$ and $n < n'$), yielding a causal dependency in each case.

\item By design choice, as made explicit in the sequel in all examples, $0 \leq \frac{\partial \Phi_{ndt}()}{\partial x_{n'}(t')} \leq 1$ (non-decreasing contractive non-linearity), is verified. This constraint is not mandatory, but will help at the numerical conditioning level.

\item We further assume, just for the sake of simplicity, that initial conditions are equal to zero, i.e., ${\bf x}(t) = 0, t < 0$ and ${\bf i}(s) = 0, s < 0$. 

\item We also assume that the dynamic is regular enough\footnote{Here, we assume that input and output are bounded, while the system is regular enough for the subsequent estimation to be numerically stable. Chaotic behaviors likely require very different numerical methods (taking explicitly the exponential dependency on previous value variations into account) \cite{cessac_view_2010}. In practice, not only contracting systems can be considered, as soon as the observation times are not too large with respect to cumulative rounding errors. As far as computing capabilities are considered, systems at the edge of chaos (but not chaotic) seem to be interesting to consider \cite{bertschinger-natschlager:04,Legenstein:2007}, which fits with the present requirement.} for weight estimation to be numerically stable.

\end{itemize}

The key point here, is that we have introduced intermediate internal state variables in order the weight estimation to be a simple linear problem as a function of these additional variables (and at the cost of higher dimensional problem).

The claim of this paper is that this choice of notation has two main consequences developed in the next sections: \begin{enumerate}
\item All known computational networks architecture can be specified that way.
\item The weight estimation problem writes in a quite simple way, with this reformulation.
\end{enumerate}

This will thus help us to revisit the recurrent weight estimation problem.


\section{Recurrent weight estimation}

We implement the recurrent weight estimation as a variational problem, i.e. define: \begin{equation}\label{eq-criterion} {\bf W} = \mbox{arg min}_{{\bf W}} \mbox{min}_{\bf x} \mbox{max}_{\bf \varepsilon} \; {\cal L}({\bf W} , {\bf x}, {\bf \varepsilon}) ,\end{equation}
for adjustable network parameters or weights ${\bf W}$, given state values ${\bf x}$ and auxilary variables ${\bf \varepsilon}$, writing:
\eqline{\begin{array}{rlll} {\cal L}({\bf W} , {\bf x}, {\bf \varepsilon}) \deq
& \rho(\cdots, x_n(t), \cdots)  & \mbox{\small desired values} \\
+& \sum_{nt} \varepsilon_{nt} \, (\tilde{x}_n(t) - x_n(t)) & \mbox{\small network dynamic constraint} \\
+& {\cal R}({\bf W}) & \mbox{\small regularization} \\
\end{array}}
where $\tilde{x}_n(t)$ is a shortcut for equation~(\ref{eq-recurrent}):
\eqline{\left\{ \begin{array}{rcl}
\tilde{x}_n(t) &\deq& \Phi_{n0t}\left(\cdots, x_{n'}(t'), \cdots, i_{m}(s), \cdots\right) \\
 &+& \sum_{d = 1}^{D_{n}} W_{nd} \, \Phi_{ndt}\left(\cdots, x_{n'}(t'), \cdots, i_{m}(s), \cdots\right)
\end{array} \right.}
while $\varepsilon_{nt}$ are Lagrange multipliers, and in most of the case\footnote{More precisely, here, in the deterministic case, a simple additive criterion is used, while this is not the case for statistical criterion, as further discussed in appendix~\ref{application} and~\ref{stochastic}.} we use:
\eqline{\rho(\cdots, x_n(t), \cdots) \deq \sum_{nt} \rho_{nt}(x_n(t)).}

Here $\rho()$ is a cost-function (acting both as supervised or unsupervised variational term) and ${\cal R}({\bf W})$ some regularization term, as made explicit in Appendix~\ref{application}. The cost function includes both the term attached to the data, i.e., the fact that output values have a desired values, and regularization. These ingredients can be used to get the approximate desired output, yield sparse estimation, reduce artifact influence, obtain activity orthogonality, etc (see Appendix~\ref{application} for details).

In a nutshell, $\rho()$ and ${\cal R}$ allows one to specify the estimation problem, as a function of the unknows ${\bf W}$, ${\bf x}$ and ${\bf \varepsilon}$. Stating the estimation this way, leads us to a simplified form of the Pontryagin's minimum principle, well-known in control theory \cite{astrom:83}, and reviwed in the next section. In short, the effective related solution is derived from the normal equations of the proposed criterion.

This formulation is not new and has been formalized, by, e.g. \cite{cun_theoretical_1988}. Here we restate it at a higher level of generality, with two new aspects: (i) making explicit the role of the Lagrange multiplier (also called adjoint state in this context) for hidden units and (ii) proposing a 2nd order local estimation mechanism. The relation with other recurrent weights estimation methods is discussed in Appendix~\ref{backpropag}.

Applying standard derivations, the criterion gradient writes:
\eqline{\begin{array}{rcll}
\partial_{\varepsilon_{nt}} \, {\cal L} &=& \tilde{x}_n(t) - x_n(t) \\
&&\\
\partial_{x_{n'}(t')} \, {\cal L} &=&  
-\varepsilon_{n't'} + \rho'_{n't'} + \sum_{nt, \begin{array}{c} t' < t \leq t' + R \\ or\; t' = t, n < n'\end{array}} \beta_{nt}^{n't'} \, \varepsilon_{nt} \\
&&\\
\partial_{W_{nd}} \, {\cal L}  &=& \sum_{n'', W_{n''d} = W_{nd}} \sum_t \phi_{n''dt} \, \varepsilon_{n''t} + \partial_{W_{nd}} \, {\cal R} \\
\end{array}}
writing : 
{\small \eqline{\left\{ \begin{array}{rclcl}
\rho'_{nt} &\deq& \partial_{x_n(t)} \rho(\cdots, x_n(t), \cdots) \\
&\\
\phi_{ndt} &\deq& \Phi_{ndt}\left(\cdots, x_{n'}(t'), \cdots, i_{m}(s), \cdots\right) &=& \partial_{W_{nd}} \tilde{x}_n(t)\\
&\\
\beta_{nt}^{n't'} &\deq& \partial_{x_{n'}(t')} \phi_{n0t} + \sum_{d = 1}^{D_{n}} W_{nd} \, \partial_{x_{n'}(t')} \phi_{ndt} &=& \partial_{x_{n'}(t')} \tilde{x}_n(t) \\
&\\
\end{array} \right.}}

The sum $\sum_{nt, \begin{array}{c} t' < t \leq t' + R \;or\; t' = t, n < n'\end{array}}$ encounters for previous values and subsequent node values. This sum includes terms with $\beta_{nt}^{n't'} \neq 0$, i.e. terms for which there is a recurrent connection from the node of index $n$ at time $t$ onto the node of index $n'$ at time $t'$. We simply write $\sum_{nt}$ in the sequel, without any risk of ambiguity.

The sum $\sum_{n'', W_{n''d} = W_{nd}}$ encounters for weight sharing, i.e., the fact that weights from different units may be constrained to have the same value. We will simply write $\sum_{n''}$ in the sequel, without any risk of ambiguity.

Let us now review and discuss how we can implement such a minimization.

\subsection*{The minimization steps}

\subsubsection*{Forward simulation}

The equation $\partial_{\varepsilon_{nt}} {\cal L} = 0$ yields $x_n(t) = \tilde{x}_n(t)$. This simply means that $x_n(t)$ is iven by the network equation, i.e., equation~(\ref{eq-recurrent}). Since $\tilde{x}_n(t)$ depends on previous values at time $t' < t$, it provides a closed-form formula to evaluate $x_n(t)$ from the beginning to the end. This simply corresponds to the fact that the dynamic is simulated. This step depends on the weights $W_{nd}$ but not on the Lagrange multipliers $\varepsilon_{nt}$. At the end of the step the equality $\partial_{\varepsilon_{nt}} \, {\cal L} = 0$ is obtained, and the criterion value itself does not depends on $\varepsilon$ since the constraints are verified. As a consequence, the criterion value ${\cal L}$ can be calculated during this step.

The forward simulation complexity corresponds to the network simulation and is of order $O(N D T)$ with a memory resources of $O(N T)$ since we must buffer the calculated output,  for subsequent calculations.

\subsubsection*{Backward tuning}

The equation $\partial_{x_{n'}(t')} {\cal L}  = 0$ also provides a closed-form formula to evaluate $\varepsilon_{n't'}$ as a linear function of subsequent values $\varepsilon_{nt}, t > t'$, so that the calculation is to be done from the last time $t = T-1$ backward to the first time $t = 0$:
\begin{equation} \label{backward-tuning}
\varepsilon_{n't'} = \rho'_{nt} + \sum_{nt} \beta_{nt}^{n't'} \, \varepsilon_{nt}.
\end{equation}

This is the key feature of such a variational approach, allowing backward tuning, i.e., take into account the fact that adjusting the system parameters for a node $n$ at time $t$ is interdependent with the state of subsequent computations.

This makes the key difference with respect to usual approaches based on gradient back-propagation: Here the output error is back-propagated. 
This calculation may be recognized as a kind of back-propagation, but it is mathematically different.
This method is thus quite different from back-propagation-though-time recurrent network or other standard alternatives. 


As mentioned by \cite{cun_theoretical_1988}, $\beta_{nt}^{n't'}$ is nothing more than the first order approximation of the backward dynamics, technically the product of the weight matrix with the system Jacobian.

This backward computation is local to a given unit in the sense that only efferent units (i.e., units this unit is connected to) are involved in the computation of the related Lagrange parameter. This step depends on both weights and output values, and the equality $\partial_{\varepsilon_{nt}} \, {\cal L} = 0$ is obtained at the end.

The backward tuning step has the same order of magnitude in terms of calculation $O(N D T)$  and memory resources of $O(N T)$ (in fact of $O(N R)$, because the obtained result may be immediately re-used to compute the 2nd and 1st order weight adjustment quantities, discussed in the sequel). 

\paragraph{\em Parameter interpretation.}

We obtain, from equation~(\ref{backward-tuning}) after some algebra $\varepsilon_{n't'} = \sum_{nt} B_{n't'}^{nt} \, \rho'_{nt}$, with finite summations and for some quantities $B_{n't'}^{nt}$ (not made explicit here) which are unary coefficient polynomial in $\beta_{nt}^{n't'}$. This made explicit the fact $\varepsilon_{n't'}$ is a linear function of subsequent errors, i.e., a backward tuning error.

If $\beta_{nt}^{n't'} = 0$, there is no dependency of ${x}_n(t)$ on ${x}_{n'}(t')$, i.e. no recurrent connection. If the unit has no recurrent connection, i.e. is a not a function of other units, then $\varepsilon_{n't'} = \rho'_{nt}$ is simply related to the cost function derivative. In the least-square case (i.e. if $\rho_{nt} = \frac{1}{2} \, (x_{nt} - \bar{o}_{nt})^2$), then $\rho'_{nt} = x_{nt} - \bar{o}_{nt}$ is the output error.

\paragraph{\em Real-time aspects.}

Such a formulation is definitely not ``real-time'', since we ``go back in time''. It is however, the only solution for hidden layers to be tuned, since the output adjustment is a function of hidden activity in the past, the estimation must thus take future information into account in order to properly adapat.

However, in a real-time paradigm, it must be noted that each computation is also local in {\em time}: It only depends on values in a ``near future'' within a time range equal to the system time range. In other words, at a given time we obtain the value with a lag equal to system time-range. It is an interesting perspective of this work to explore if, considering only a bounded window-time may provide numerically relevant values for on-the-fly backward tuning.

\paragraph{\em Numerical stability.}

This back-propagation of tuning error, may suffer from the same curse than back-propagation of gradient, as reviewed in e.g., \cite{Hochreiter:1997}: Either error explosion (if $|\beta_{nt}^{n't'}| > 1$), or error extinction (if $|\beta_{nt}^{n't'}| < 1$). Based on this remark, the key idea of LSTM \cite{Hochreiter:1997} is to consider memory carousel (detailled in Appendix~\ref{generality}) to guaranty $\left|\beta_{nt}^{n't'}\right| \simeq 1$ and thus a stable back-propagation for at least some recurrent link, but this means that the designer of the network architecture has to consider such predefined units, which is a strong constraint.

In our case, since all kernels are contracting with $\max|\partial_{x_{n'}(t')} \phi_{ndt}| = 1$ we are in a situation where the a-priory numerical conditioning is optimal. We also have the bound, writing $\beta_{max} \deq \max_{nt} |\beta_{nt}^{n't'}|$ :
\eqline{0 \leq \left|\beta_{nt}^{n't'}\right| \leq \beta_{max} \leq 1 + \sum_d |W_{nd}|}
without any thinner inequality in the general case. This means that we ``must'' accept error potential explosion as soon as the weights values are not below one, which can not be a manageable constraint.

\begin{figure}[!ht]
  \includegraphics[width=0.8\textwidth]{img/backward-guard}
  \caption{The backward guard profile, defined in~(\ref{eq-g}), with a bias for tiny values and a saturation for huge values.}
  \label{backward-guard}
\end{figure}

To avoid backward explosion or extinction, we are going to introduce another heuristic: We are going to {\em bias} the backward error given in equation~(\ref{backward-tuning}). We define:
\begin{equation} \label{kappa}
\varepsilon_{n't'} \simeq \rho'_{nt} + g\left(\sum_{nt} \beta_{nt}^{n't'} \, \varepsilon_{nt}\right),
\end{equation} considering a function $g(u)$, shown in Fig.~\ref{backward-guard}. It is the identity function except for small vanishing values that are raised using a simple quadratic profile, an huge values saturated by an exponential profile, and providing a continuously derivable function. This design choice writes, for fixed meta-parameters $\omega, \alpha, \nu$:
\begin{equation}\label{eq-g} g(u) \deq sg(u) \, \left\{ \begin{array}{cl} 
\omega - \alpha \, e^{-\frac{|u| - \omega}{\alpha} - 1} & \omega - \alpha \leq |u| \\
|u| & 2 \, \nu \leq |u| \leq \omega - \alpha \\
\nu + \frac{u^2}{4 \, \nu} & |u| \leq 2 \, \nu, \\
\end{array} \right.\end{equation}
where $sg()$ is the sign function. To fix these meta-parameters we consider the order of magnitude of the output error:
\eqline{\bar{\rho'} \deq \frac{\sum_{nt, \rho'_{nt} \neq 0} \rho'_{nt}}{\sum_{nt, \rho'_{nt} \neq 0} 1}, }
and a reasonable choice to preserve the numerical conditioning is $\nu = 10^{-6} \, \bar{\rho'}$ and $\omega = 10^{6} \, \bar{\rho'}$, with e.g., $\alpha = 10^{-3} \, \omega$. They are very likely not to be adjusted because they only correspond to order of magnitude of numerical calculation. We have observed that using double precision floating numbers on a standard processor for such kind of calculations corresponds to such rough numbers.

\subsubsection*{The 2nd order unit weight adjustment}

We now have to estimate the weights ${\bf W}$ and are left with the last normal equation $\partial_{W_{nd}}\, {\cal L} = 0$ which is not an explicit function of the weights. On track is to use the gradient to minimize the criterion using a 1st order method, this is discussed in the next sub-section. Interesting enough is the fact that we can also propose a 2nd order method as made explicit and derived now. In other words, we reintroduce a linear estimation of the weights assuming that the criterion is locally quadratic.

We thus propose to use the following 2nd order weight adjustment:
\begin{equation} \label{2nd-order}
\sum_{n''} b_{n'',\,d} = \sum_{n''} \sum_{d'=1}^{D_{n}} A_{n'',\,d\;d'} \, W_{n''d'}
\end{equation}
writing, for some $\kappa_{nt}$: 
\eqline{\left\{ \begin{array}{rcl}
b_{n,\,d} &\deq& \sum_t \phi_{ndt} \, \left( \varepsilon_{nt}  + \kappa_{nt} \, \left(\hat{x}_n(t) - \phi_{n0t} \right) \right)  + \partial_{W_{nd}} \, {\cal R}(\hat{\bf W}),  \\
&\\
A_{n,\,d\;d'} &\deq& \sum_t \kappa_{nt} \, \phi_{ndt} \, \phi_{nd't},  \\
\end{array} \right.}
where:
\\- $\hat{x}_n(t)$ is best present estimate of $x_n(t)$,
\\- $\hat{\bf W}$ is the best estimate of ${\bf W}$ at the present step.

This allows us to obtain a new weight value ${\bf W}$ solving a linear system of equation for each unit and the closest solution\footnote{{\bf Minimal distance pseudo-inverse.} We consider: 
\eqline{\min_{{\bf W}} ||{\bf W} - \hat{\bf W}||, {\bf b} = {\bf A} \, \hat{\bf W}}
which is directly obtained using the singular value decomposition of the symmetric matrix ${\bf A} = {\bf U} \, {\bf S} \, {\bf U}^T$:
\eqline{{\bf W} = \hat{\bf W} + {\bf A}^\dagger \, ({\bf b} - {\bf A} \, \hat{\bf W}),} 
where ${\bf A}^\dagger$ is the pseudo-inverse of ${\bf A}$.\hr} with respect to $\hat{\bf W}$ is considered.

The derivation\footnote{\label{improvingkappa}{\bf Deriving the 2nd order adjustment form.} Let us omit the ${\cal R}()$ term and avoid considering weight sharing in this derivation, in order to lighten the notations.  The complete derivation would have obviously led to similar results.

Given a desired value estimate $\hat{x}_n(t)$, without loss of generality we can write, for some general quantity $\kappa_{nt}({\bf W} , {\bf x}, {\bf \varepsilon})$:
\eqline{{\cal L}({\bf W} , {\bf x}, {\bf \varepsilon}) = \sum_{nt} \frac{\kappa_{nt}({\bf W} , {\bf x}, {\bf \varepsilon})}{2} \, \left(\tilde{x}_n(t) - \hat{x}_n(t)\right)^2,}
yielding for $\partial_{W_{nd}} {\cal L}$:
\eqline{\sum_{t} \kappa_{nt}({\bf W} , {\bf x}, {\bf \varepsilon})  \, \phi_{ndt} \, (\tilde{x}_n(t) - \hat{x}_n(t)) + \sum_{t} \partial_{W_{nd}} \kappa_{nt}({\bf W} , {\bf x}, {\bf \varepsilon}) \, (\tilde{x}_n(t) - \hat{x}_n(t))^2 / 2 = \sum_t \phi_{ndt} \, \varepsilon_{nt}.} 
For a simple least-square criterion,  $\kappa_{nt} \in \{0, 1\}$ depending on the fact that the desired output $\bar{o}_n(t)$ is defined or not, and it is straight-forward to verify in this particular case that the proposed 2nd order weight adjustment reduces to an exact linear system of equation, in the absence of recurrent links of the given unit, since $\phi_{ndt}$ is only function of the input. Otherwise, $\phi_{ndt}$ is also a function of both the network unknown output and hidden node values.
\\ (i) In our case the output and backward error estimation is $\varepsilon_{nt}$, i.e., we can set $\hat{x}_n(t) \deq \tilde{x}_n(t) - \varepsilon_{nt}$ as a corrected value of the last estimate $\tilde{x}_n(t)$. Given this hypothesis, it is obvious to verify that $\kappa_{nt}({\bf W} , {\bf x}, {\bf \varepsilon}) = 1$ verifies the equation.
% dsolve({kappa(w) * phi * (phi * w - psi) + D(kappa)(w) * (phi * w - psi)^2 / 2 = phi * epsilon}, {kappa(w)}); 
\\ (ii) A step further, for a general value $\hat{x}_n(t)$, a sufficient condition now writes:
\eqline{\kappa_{nt}({\bf W} , {\bf x}, {\bf \varepsilon}) = 2 \, \varepsilon_{nt} / (\tilde{x}_n(t) - \hat{x}_n(t)),}
thus $k_nt$ is proportional to $\varepsilon_{nt}$ and must decrease with the prediction error increase.
Considering the case (i), it is straightforward to verify that if $\kappa_{nt}$ is constant, and assuming $\hat{x}_n(t)$ is fixed, we obtain the related least-square linear equations given in~(\ref{2nd-order}).
\hr}.

More sophisticated estimated values can be considered\footnote{\label{improvingstate}{\bf Improving the best estimate of the state value.} The best estimate of the state value $\hat{x}_n(t)$ given output values $\bar{o}_{n_0}(t)$ is not obtained by the simulation since $\hat{x}_{n_0}(t) \neq \bar{o}_{n_0}(t)$.

If we consider the value obtained by simulation (i.e., the $\tilde{x}_n(t)$ values), corrected by the error estimate thus $\hat{x}_n(t) = \tilde{x}_n(t) - \varepsilon_{nt}$, for a least-square criterion, it is easy to verify that this yields $\hat{x}_{n_0}(t) = \bar{o}_{n_0}(t)$. 

For output node value the $\bar{o}_n(t)$ desired value could be enforced, limiting recurrent perturbation and yielding $\phi_{ndt}$ values closed to the ideal value, which is interesting in reverse-engineering estimation, i.e. when an exact solution is expected \cite{rostro-gonzalez-cessac-etal:10}, whereas a bias in the estimation is otherwise expected, since hidden units simulated values and output values are not coherent.

A step further, we propose to retro-propagate the output value through the recurrent network, given weights values $\hat{\bf W}$, i.e., estimate:
\eqline{\hat{x}_n(t) = \mbox{arg min}_{{x}_n(t)} {\cal M}, \;\;\; {\cal M} = \frac{1}{2} \sum_{n, n \geq N_0 \; t} ({x}_n(t) - \Phi_{nt}\left(\cdots, {x}_{n'}(t'), \cdots\right))^2, {x}_{n_0}(t) = \bar{o}_n(t)}
in words find the state values for which the simulation errors yielding the desired output are minimal. Considering the normal equation $\partial_{x_{n'}(t')} {\cal M} = 0$ we obtain the recurrent equation:
\eqline{\hat{x}^k_{n'}(t') = \left\{\begin{array}{ll}
 \bar{o}_{n'}(t') & n' < N_0 \\
 \Phi_{nt}\left(\cdots, \hat{x}^{k-1}_{n'}(t'), \cdots\right) - \sum_{nt} \beta_{nt}^{n't'} \, (\hat{x}^{k-1}_{n}(t) - \Phi_{nt}\left(\cdots, \hat{x}^{k-1}_{n'}(t'), \cdots\right)) & N_0 \leq n', \end{array}\right.}
i.e., the simulation value is corrected considering a backward propagation of the simulation error.

In fact, it is to verify that we implicitly solve a system of $N\,T$ equations in $N\,T$ unknowns, the numerical scheme allowing to converge to a solution closed to the simulation values. This has been numerically verified in the experimentation.

It has been implemented as an option in the software in order to help improving the convergence of the recurrent weight adjustment. \hr}.

The weight adjustment is local to each unit, providing a true distributed mechanism  (unless if weight sharing is considered, because weights from different units are to be estimated together using the proposed equations). This corresponds to a 2nd order minimization scheme. Each step requires $O(N (D T + D^3)$- operation, solving a linear system of equations. The $O(N\,D^3)$ is critical if the network connectivity $D$ is high, and this does not depend on the linear system resolution method (e.g., SVD or Cholesky decomposition). The implemented method stands on the singular-value-decomposition of the matrices ${\bf A}_n$.

This offers an alternative to 2nd order adjustment methods such as \cite{martens_learning_2016} or other methods reviewed in \cite{Goodfellow2016Deep}.

In fact, a standard 2nd order adjustment can be derived in closed form\footnote{{\bf Calculating the standard 2nd order weight adjustment.} The criterion Hessian, omitting the regularization term and weight sharing to lighten the notations, writes:
\eqline{\left\{\begin{array}{rclcl}
\partial_{\varepsilon_{nt} \varepsilon_{n't'}} \, {\cal L} &=& 0 \\
\partial_{x_{n'}(t') \varepsilon_{nt}} \, {\cal L} &=& \beta_{nt}^{n't'} \\
\partial_{W_{nd} \varepsilon_{n't'}} \, {\cal L} &=& \delta_{n=n'} \, \phi_{ndt'} \\
\partial_{x_{n'}(t') x_{n''}(t'')} \, {\cal L} &=&  H^{nt}_{n't'n''t''} &\deq& \sum_{nt} \partial_{x_{n'}(t') x_{n''}(t'')} \left(\rho() + \phi_{n0t} + \sum_{d = 1}^{D_{n}} W_{nd} \, \phi_{ndt}\right) \\
\partial_{W_{nd} x_{n'}(t')} \, {\cal L} &=&  J^{nd}_{n't'} &\deq& \sum_t \varepsilon_{nt} \, \partial_{x_{n'}(t')} \phi_{ndt} \\
\partial_{W_{nd} W_{n'd'}} \, {\cal L} &=& 0, \\
\end{array}\right.}
writing $\beta_{nt}^{nt} \deq -1$. 

The 1st remark is that $H^{nt}_{n't'n''t''}$ and $J^{nd}_{n't'}$ are not local to one node, whereas the summation involves all nodes connected to the given one. Furthermore if $\rho()$ is not a sum of local terms but a statistical criterion $H^{nt}_{n't'n''t''}$ is a function of the whole network.

Then the standard 2nd order scheme $0 \simeq \nabla {\cal L} + \nabla^2 {\cal L} \, \delta({\bf W}, {\bf x}, {\bf \varepsilon})$ writes in our case where $\partial_{\varepsilon_{nt}} {\cal L} = \partial_{x_{n'}(t')} {\cal L} = 0$:
\eqline{\left\{\begin{array}{rccrccrccl}
&&&\sum_{n't'} \beta_{nt}^{n't'} & \delta x_{n'}(t') &+& \sum_d \phi_{ndt} & \delta W_{nd}  &\simeq& 0_{nt} \\
\sum_{nt} \beta_{nt}^{n't'} & \delta \varepsilon_{nt} &+& \sum_{n''t''} H^{nt}_{n't'n''t''} & \delta x_{n''}(t'') &+& \sum_{nd} J^{nd}_{n't'} & \delta W_{nd}   &\simeq& 0_{n't'} \\
\sum_t \phi_{ndt} & \delta \varepsilon_{nt} &+& \sum_{n't'} J^{nd}_{n't'} & \delta x_{n'}(t')  &+& \sum_t \phi_{ndt} \, \varepsilon_{nt} &&\simeq& 0_{nd}, \\
\end{array}\right.}
and $\delta \varepsilon_{nt}$ and $\delta x_{n'}(t')$ can be eliminated in order to obtain a linear equation in $\delta W_{nd}$. This however requires the inversion the $\beta_{nt}^{n't'}$ matrix (and its transpose), which is a $O(N\,T \times N\,T)$ matrix, not necessarily sparse if the network is fully connected. We thus consider that the resulting calculation is too greedy to be performed at each step of the minimization.\hr}, directly from the 2nd order criterion derivatives. It is not used here because the computation involves not only the local node parameters, but also the connected node parameters, and the calculation is rather heavy.

\subsubsection*{The 1st order unit weight adjustment}

The calculation of $\partial_{W_{nd}}\, {\cal L}$ allows us to propose a 1st order gradient descent adjustment of the weights, providing that $\partial_{\varepsilon_{nt}} {\cal L} = 0$ after network simulation and $\partial_{x_{n'}(t')} {\cal L} = 0$ after backward tuning.

It yields a Hebbian weight adaptation rule (as the sum of products between an output unit error term $\varepsilon_{nt}$ (combining the supervised error and the backward tuning multiplier) and an input quantity $\phi_{ndt}$. This rule applies to both output unit of index $n < N_0$ with a desired output and hidden units of index $N_0 \leq n$ that indirectly adapt their behavior to optimize the output, via the backward tuning values. The gradient calculation is local to a given unit and average over time, through another $O(NDT)$ computation, unless weight sharing is considered. In that case, this 1st order unit weight adjustment is either to be done globally at the whole node set level, or locally for each unit, but with inter-unit weight adjustment, not discussed here. 

A step further, we can enhance this method considering the so-called momentum gradient mechanism (based on a temporal averaging of the gradient values). To this end we consider:
\eqline{{\bf g}_k(t) = (1 -  1 / k) \, {\bf g}_k(t-1) + 1/k \, \partial_{W_{nd}}\, {\cal L}(t), \;\;\; k \in \{1, 2, 4, 8, 16, 32\}}
in words, a 1st order exponential filtering of the gradient value obtained at time $t$, and the algorithm is going to compare these 6 options and choose the one with a maximal criterion decrease (avoiding introducing a meta-parameter at this stage). Here we mainly would like to explore several direction of descent if the criterion is not numerically regular.

This leads to a 1st order adjustment of the weights, i.e. it provides the direction for the weight variation, not its magnitude. In order to manage this issue we very simply automatically adjust a step meta-parameter $\upsilon^k$, initialized to any reasonnable small value and:\begin{quotation}{\small \noindent
- Calculates: $\tilde{W}_{nd} = \hat{W}_{nd} - \upsilon^k \, {\bf g}_k$.
\\\hspace{0.5cm} - Performs a rough line-search minimization $\min_{\alpha^k} {\cal L}(\alpha^k \tilde{\bf W} + (1 - \alpha^k) \, \hat{\bf W})$ (here using the Brent-Dekker method with a $10^{-2}$ relative precision).
\\\hspace{0.5cm} - Updates $\upsilon^k \leftarrow 2 \, \alpha^k \, \upsilon^k$.}\end{quotation}
In words we look for a weight value between both previous and new values that decreases the criterion, and set the new step value to twice the last optimal value. Each line-search step requires a simulation to compute ${\cal L}$. 

This is a bit heavy, but it is only a fall-back of the 2nd order adjustment (e.g., for concave parts of the criterion). For the same reason, more sophisticated methods such as conjoint gradient methods (taking into account several subsequent gradient directions in order to infer an approximate 2nd order minimization method) have not been considered.

\subsubsection*{The complete weight adjustment}

\begin{figure}[!ht]
  \includegraphics[width=0.8\textwidth]{img/estimation-algo}
  \caption{The algorithmic structure of the estimation algorithm: A forward simulation yields the current criterion value, while the backward tuning allows us to obtain the 2nd order and 1st order local weight adjustment elements. The algorithm can be implemented in a complete distributed framework.}
  \label{estimation-algo}
\end{figure}

Collecting the previous steps the final iterative weight adjustment writes\begin{quotation}{\small 
\noindent -1- Perform a forward simulation and a backward tuning, calculating the 1st order gradient and 2nd order elements during the backward estimation.
\\\hspace{0.5cm} -2.a- Perform a 2nd order weight adjustment.
\\\hspace{0.5cm} -2.b- If it fails, attempt to perform a 1st order weight adjustment.
\\-3- Repeat -1- unless steps -2.b- fails.
}\end{quotation}

The 2nd order adjustment also uses a line search, because our experimental observation is that the 2nd order estimation tends to overestimate the local minimum. The 2nd order adjustment is not performed if the connectivity of the network is too high since it has a cubic cost.

Though the algorithm can be implemented in a complete distributed framework, in this preliminary study, the 2nd or 1st order adjustment is global, in order to limit the number of iteration on a simple sequential machine. The complete algorithmic structure is schematized in Fig.~\ref{estimation-algo}.


\clearpage

\section*{Experimentation}

In this experimental part we are going to both study the numerical stability and limit of the method and compare to existing non-trivial benchmark problems. Here, supervised learning is targeted since it is a direct way to evaluate the method efficiency and robustness. Let us remember that we do not evaluate learning performances here, only the way me can adjust recurrent network weights. We only study the estimation convergence here, not the learning properties (such as generalization, capacity, ...).

\subsection*{Software implementation}

In order to provide so called reproducible science \cite{topalidou_long_2015}, the code is implemented as a very simple, highly modular, fully documented, open source, object oriented, easily forkable, self contained, middle-ware, \href{nowhere:not-yet-on-line}{available here}. A minimal set of standard mechanisms (random number generation, histogram estimation, linear system resolution, system calls) is used. The main part of the implementation hierarchy is show in Fig.~\ref{class-hierarchy}. 

\begin{figure}[!ht]
  \includegraphics[width=0.8\textwidth]{img/class-hierarchy}
  \caption{A view of the class-hierarchy: A {\tt Input} simply provides a $x_n(t), n \in \{0, N\{, t \in \{0, T\{$ values, while a {\tt Transform} provides such values given another {\tt Input}, while other objects defined here derive from such an oversimple abstract class.}
  \label{class-hierarchy}
\end{figure}

For run-time performances and inter-operability with different programming languages a {\tt C/C++} implementation (with the compilation scripts) is proposed, the wrapping to other programming languages (e.g., {\tt Python}) being straightforward, using e.g. {\tt swig}. 

The first experimental verification was that it was very simple to define the main unit structures reviewed in section~\ref{generality} from {\tt KernelTransform} as claimed in the paper.

\subsection*{Numerical stability and limit of the method}

Regarding this first issue, as being in a deterministic context, we are going to rely on a reverse engineering setup: An input/output learning sequence is going to be generated by a input/output root network of $\bar{N}$ units and another learning network with random initialization is going to re-estimate a transform. This guaranty the existence of an exact solution. 

How revelant is it to use such a reverse engineering setup ? On one hand, surprisingly enough perhaps, such networks (at least deep networks \cite{Zhang2016Understanding}) behave with the same order of magnitude of performances, the input being either ``meaningful'' in the sense it represents data with a semantic or not. We thus can expect simple random input/output tests to be relevant to more specific application. On the other hand we are going to develop in the next section how several ``challenging'' tests are in fact highly dependent on the chosen architecture, with often trivial solution, as soon as the hidden architecture is well chosen.

Considering random input/output with some statistics, in most of the cases, they are several solutions (e.g. up to a permutation of the units, or some linear combination in a linear case, ...). We consider a root network of $\bar{N}$ units for a sequence of time $T$, for a $M=1$ scalar input, considering either L (for linear), LNL or AIF units, with random weights (drawn from a Gaussian distribution with $0$ mean and $\sigma \simeq 1/N$ standard deviation, which is know to guaranty a stable non-trivial dynamic). Only the unit of index $n=0$ is considered as output units, i.e., $N_0 = 1$, the $N-1$ remainder units activity being hidden to the estimation.

In this determistic case, we observe two main parameters the final precision criterion value ${\cal L}$ and the number of steps to convergence $S$. 

A step further, we also study exact solution or approximation, we are going to consider learning network with $N \leq \bar{N}$ units, i.e. networks that do not generate the exact solution. We already know that as soon as the dynamic is sufficiently rich, even small errors accumulates and the solution exponentially diverges from the exact one. In such a case the question is wether the input/output statistics also diverges. We thus have to compare the KL-divergence between the desired and obtained output given the input. The fact we chose $N_0 = 1$ makes this estimation tracktable since it is a 1D distribution. OUI MAIS JUSTE HISTOGRAMME QUID DE LA DEPENDENCE EN i(t-R) ?

\subsection*{Comparison with existing estimation problems}

Let us now discuss how our validation method compares with usual benchmark for recurrent network estimation.

\section{Conclusion}We consider another formulation of weight estimation in recurrent networks, proposing a notation for a large amount of recurrent network units that helps formulating the estimation problem. Reusing a ``good old'' control-theory principle, improved here using a continuation heuristic, we obtain a numerically stable and rather efficient second-order and distributed estimation, without any meta-parameter to adjust. The relation with existing technique is discussed at each step. 
The proposed method is validated using reverse engineering tasks and non-trivial numerical computations.


\appendix
\clearpage\section{Major examples fitting this architecture.} \label{generality}

The notation of equation~(\ref{eq-recurrent}) seems to be the most general form of usual recurrent networks. Let us state this point by considering several examples of units, and make explicit how we decompose them in term of nodes.

\paragraph{Linear non-linear (LNL) units.} 

Such network unit corresponds to the most common\footnote{See also a dual form related to AIF, in the sequel, with an alternate insertion of the non-linearity.} network unit and is defined by a recurrent equation of the form:
\begin{equation}\label{lnl-network}
\begin{array}{rcl}x_n(t) &=& \gamma_n \, x_n(t-1) \\ &+& \zeta_{[a,b]}\left(\alpha_n + \sum_{n' = 0}^{N-1} W_{nn'} \, x_{n'}(t-1) + \sum_{m = 0}^{M-1} W_{nm} \, i_m(t-1)\right), \end{array}\end{equation}
\\- with either a fixed or adjustable {\em leak}\footnote{Here $\gamma = 1 - \frac{\Delta T}{\tau}$ stands for the leak of each unit, writing $\Delta T$ the sampling period, $\tau$ the continuous leak and using an basic trivial Euler discretization scheme, the $\zeta()$ profile being re-normalized accordingly.} $\gamma_n$, providing $0 < \gamma_n < 1$, and 
\\- optionally {\em intrinsic plasticity} parameterized by $\alpha_n$. 

The {\em non-linearity} often\footnote{If the model corresponds to a rate, i.e., a firing probability, we can use the logistic sigmoid, which writes $\zeta_{[0,1]}(u) = \frac{1}{1 + e^{-4 \, u}} = \frac{1 + \tanh(2 \, u)}{2}$.} writes
\eqline{\zeta_{[a,b]}(u) \deq \frac{a+b}{2} + \frac{b-a}{2} \, \tanh(\frac{2}{b-a} \, u),}
with $\zeta_{[a,b]}(-\infty) = a, \zeta_{[a,b]}(+\infty) = b, \zeta_{[a,b]}(u) = \frac{a+b}{2} + u + O(u^3)$,  while $\zeta'(u) = 1 - \tanh(\frac{2}{b-a} \, u)^2$,  $0 < \zeta'(u) \leq 1$, with $\max|\zeta'(u)| = 1$, thus contracting with a correct numerical conditioning. We mainly have $[a,b] = [0,1]$ or $[a,b] = [-1,1]$ depending on the semantic interpretation of the $x_n(t)$ variable.

Another form of non-linearity is a rectified linear unit (or ReLU), i.e.:
\eqline{\zeta_{[0,+\infty]}(u) \deq \max(0, u).}
This function is not derivable at $u = 0$. It is however very easy to consider a mollification (called ``softplus'') e.g., $\zeta_{\epsilon, [0,+\infty]}(u) \deq \epsilon \, \log\left(1 + e^{\frac{u}{\epsilon}}\right)$ which is an analytic smooth approximation which uniformly converges\footnote{Since $\forall u, |\zeta_{\epsilon, [0,+\infty]}(u) - \zeta_{[0,+\infty]}(u)| \leq \log(2) \, \epsilon$.}, i.e. $\lim_{\epsilon \rightarrow 0} \zeta_{\epsilon, [0,+\infty]}(u) = \zeta_{[0,+\infty]}(u)$. See the section on AIF units to see how to adjust, if needed, such a meta-parameter redefining it as a node parameter.

For adjustable leak we need three nodes to fit within the proposed notations:
\eqline{\begin{array}{rcl}
 x_n(t) &=& x_{n_1}(t) + \zeta_{[a,b]}\left(x_{n_2}(t)\right) \\
 x_{n_1}(t) &=& \gamma_n \, x_{n}(t-1) \\
 x_{n_2}(t) &=& \alpha_n + \sum_{n' = 0}^{N} W_{nn'} \, x_{n'}(t-1) + \sum_{m = 0}^{N} W_{nm} \, i_m(t-1) \\
\end{array}}
and it is easy to verify that this second form fits with equation~(\ref{eq-recurrent}), since:
\\- The 1st line corresponds to a parameter-less $\Phi_{n0t}\left(\right)$ kernel (unit firmware).
\\- The 2nd and 3rd lines correspond to linear combinations of elementary kernels $\Phi_{ndt}\left(\right)$ selecting another state or input variable (unit learnware).

With this example, we see that the proposed approach is to introduce two additional intermediate variables $x_{n_1}(t)$ and $x_{n_2}(t)$ related to each linear combination of weights or other parameter.

With a fixed leak (i.e., if the value $\gamma_n$ is known) the LNL unit decomposes into two nodes, a parameter less node combining $x_n(t)$ and $x_{n_1}(t)$, and the linear combination defined for $x_{n_2}(t)$.

This equation is also valid for the main auto-encoder architectures, and for convolution networks \cite{Bengio:2009,Deng:2014}, with an important additional feature : weight-sharing, i.e. the fact that several weights $W_{nd}$ are the same across different nodes. This is taken into account in this paper.

\paragraph{Long short term memory (LSTM) units.} 

Such network unit is defined by a sophisticated architecture \cite{Hochreiter:1997}, described in figure~\ref{recurrent-network}. A unit is made of the following nodes:
\begin{equation}\label{lstm-network}
\begin{array}{rcl}
\multicolumn{3}{l}{\mbox{Unit output:}} \\
x_n(t) &=& \zeta_{[0,1]}\left(y_n^{out}(t)\right) \, \zeta_{[-1,1]}\left(s_n(t)\right) \\
\multicolumn{3}{l}{\mbox{Unit state:}} \\
s_n(t) &=& \zeta_{[0,1]}\left(y_n^{forget}(t)\right) \, s_n(t-1) + 
          \zeta_{[0,1]}\left(y_n^{in}(t)\right) \, \zeta_{[-1,1]}\left(g_n(t)\right) \\
\multicolumn{3}{l}{\mbox{Unit gate:}} \\
g_n(t) &=& \sum_{n'} W^g_{nn'} \, x_{n'}(t-1) + \sum_{m} W^g_{nm} \, i_m(t-1) \\
\multicolumn{3}{l}{\mbox{Output modulation:}} \\
y_n^{out}(t) &=& W^o_n \, s_n(t-1) + \sum_{n'} W^o_{nn'} \, y_{n'}^c(t-1) + \sum_{m} W^o_{nm} \, i_m(t-1) \\
\multicolumn{3}{l}{\mbox{Forgetting modulation:}} \\
y_n^{forget}(t) &=& W^f_n \, s_n(t-1) + \sum_{n'} W^f_{nn'} \, y_{n'}^c(t-1) + \sum_{m} W^f_{nm} \, i_m(t-1) \\
\multicolumn{3}{l}{\mbox{Memorizing modulation:}} \\
y_n^{in}(t) &=& W^i_n \, s_n(t-1) + \sum_{n'} W^i_{nn'} \, y_{n'}^c(t-1) + \sum_{m} W^i_{nm} \, i_m(t-1)  \\
\end{array}\end{equation}

The first two nodes are parameter-less additive and/or multiplicative combination of non-linear functions of the reminding four nodes, which are themselves linear combination of the incoming signal gate and the input, forgetting and output modulatory signals. 

The present notation corresponds to the most general form (e.g., with peephole connections \cite{Gers:2003}) of LSTM, while several variants exist. A rather closed mechanism is named gate recurrent unit \cite{Cho-2014}, and is based on the same basic ideas of modulatory combination, but with a simpler architecture. We do not make explicit the equations for all variants of LSTM here, just notice that they correspond to some of the very best solutions for high performance recurrent network computation \cite{Schmidhuber:2015}.

\begin{figure}[!ht]
  \includegraphics[width=0.8\textwidth]{img/lstm-unit}
  \caption{A LSTM unit has three processing stages for bottom to top: The (i) gate $g$ corresponds to as standard LNL unit that (ii) feeds an internal state memory $s$ which value is also driven by a forget (or remember) signal allowing to maintain the previous value, before (iii) the output connected value $x$ diffuse (or not) the result in the network. The LSTM mechanism is thus based on three ingredients, (a) the use of modulatory connection (i.e., with a multiplication by a number between 0 and 1 in order to control the signal gain), (b) a memory ``carrousel'' (i.e., an equation that could be of the form $s_n(t) = s_n(t-1)$ in order to maintain a signal, during a long short-term delay), and (c) the use of several modulatory signals. From \cite{Hochreiter:1997}.}
  \label{lstm-unit}
\end{figure}

However, in our context, instead of reusing such a complex unit as it, the design choice is to consider the non standard nodes (i.e., unit output and unit state) as modular nodes that could be combined with NLN at different level of complexity, depending on the task. At the implementation level we are not going to provide LSTM units as black boxes but an object-oriented framework allowing to adjust the network architecture to the dedicated task.

A key-point is that LSTM have, by construction, a real virtue regarding weight adjustment since back-propagation curses (vanishing or explosion) is avoided \cite{Schmidhuber:2015}. A strong claim of this paper is that we can efficiently adjust the recurrent network weights even if we do not use (or only use) LSTM but simpler units also.

\paragraph{Strongly-Typed Recurrent Neural units.} 

This other formalism \cite{Balduzzi:2016} carefully considers the signal type in the sense of parameters of different physical origins (e.g., Volts and meter), that cannot be simply mixed. This approach allows unary and binary functions on vectorial values of the same type, transformation from one orthogonal basis to another (thus using orthogonal matrices only) and component-wise product (i.e., modulatory combination). The authors show that strongly-typed gradients better behaved and that, despite being more constrained, strongly-typed architectures achieve lower training and comparable
generalization error to classical architectures. Considering a strongly-typed LNL unit, following \cite{Balduzzi:2016} and translating in the present notation, at the same degree of generality of LNL networks, we obtain:
\begin{equation}\label{st-lnl-network}\begin{array}{rcll}
  x_n(t) &=& \zeta_{[0,1]}\left(f_n(t)\right) \, x_n(t-1) + \left(1 - \zeta_{[0,1]}\left(f_n(t)\right)\right) \, z_n(t) \\
  f_n(t) &=& \alpha_n + \gamma_n \,  x_n(t-1) \\
  z_n(t) &=& \sum_{n' = 0}^{N} W'_{nn'} \, x_{n'}(t-1) + \sum_{m = 0}^{N} W'_{nm} \, i_m(t-1) \\
\end{array}\end{equation}
The first line is the firmware combination of the unit forgetting mechanism, this value being defined in the 2nd line, while the 3rd line performs the linear combination of other network values.
It is an interesting alternative to usual approach, embedable in our notation.

\paragraph{Approximation of {\em leaky integrate and fire} (AIF), current-driven, spiking-neuron unit.}

Let us also discuss how to cope with spiking networks (see \cite{cessac-paugam-moisy-etal:10} for a general discussion on such network computational power and limit).
 Following \cite{cessac_discrete_2008} (with a tiny change of notation), we consider without loss of generality a discretized form, which writes: \begin{equation}\label{lif-network} \begin{array}{rcl}x_n(t) &=& \gamma_n \, \left(1 - \Upsilon_\epsilon\left(x_n(t-1)\right)\right) \,x_n(t-1) \\ &+& \sum_{n' = 0}^{N} W_{nn'} \, \Upsilon_\epsilon\left(x_{n'}(t-1)\right) + \sum_{m = 0}^{N} W_{nm} \, i_m(t-1), \end{array}\end{equation} 
where the unit value is over or below the spiking threshold $\theta = 1/2$ (thus spiking or not), while the reset value is $0$.

Here, as inspired from \cite{cessac_using_2012}, we propose to use $\Upsilon_\epsilon(v) \deq \zeta_{[0,1]}\left(\frac{v-1/2}{\epsilon}\right)$, as a mollification of the threshold function\footnote{Obviously, $\lim_{\epsilon \rightarrow 0, v \neq 0} \Upsilon_\epsilon(v) = \Upsilon(v)$, while $\Upsilon'_\epsilon(0) = 1/\epsilon$ and $\int_v |\Upsilon_\epsilon(v) - \Upsilon(v)| = \log(2)/2 \, \epsilon$. Here the convergence can not be uniform (since a continuous function converges towards a step function), more precisely $\sup_v |\Upsilon_\epsilon(v) - \Upsilon(v)| = 1/2$ (around $v\simeq 1/2$). 
}: 
\eqline{\Upsilon(v) \deq \left\{\begin{array}{cl} 0 & v < 1/2 \\ 1/2 & v = 1/2 \\ 1 & 1/2 < v \\ \end{array}\right..}
To avoid spurious effects when adjusting the weights, we have to find out the best minimal $\epsilon$ value for each unit.

As far as the unit architecture is concerned, it is a simple variant of LNL unit, with different kernel function, and different positioning of the non-linearity. The key point is that this so called BMS formulation fits with the present approach:
\eqline{\begin{array}{rcl}
 x_n(t) &=& \gamma_n \, \left[(1-\zeta_{[0,1]}(x_{n_1}(t))) \, x_n(t-1)\right] \\ 
  &+& \alpha_n + \sum_{n' = 0}^{N} W_{nn'} \, \zeta_{[0,1]}\left(x_{n'_1}(t-1)\right) + 
      \sum_{m = 0}^{N} W_{nm} \, i_m(t-1) \\
 x_{n_1}(t) &=& \frac{1}{\epsilon} \, \left[x_n(t-1) - \frac{1}{2} \right] \\
\end{array}}
Here $\omega\deq\frac{1}{\epsilon}$ is now a parameter to estimate, in order each unit to be a suitable approximation of a spiking activity. This differs from \cite{cessac_using_2012} where sharpness was considered as a meta-parameter: Here it is a parameter learned on the data. In both cases, we need ${\epsilon \rightarrow 0}$, which means that the transformation is very sharp, limiting the numerical stability. This is going to be investigated at the numerical level.

The use of such units is very interesting in practice and we review in appendix~\ref{closedforms} how they can be used to propose trivial solutions to rather complex tasks.

\paragraph{Softmax and exponential probability units.} 

When considering exponential distribution of probability on one hand, or softmax\footnote{
The relation with a max operator comes from the fact that:
\eqline{x_n(t) \deq \frac{e^{\frac{z_n(t)}{\epsilon}}}{\sum_n e^{\frac{z_n(t)}{\epsilon}}} \Rightarrow \lim_{\epsilon \rightarrow 0} \sum_n x_n(t) \, z_n(t) = \max_n \left( z_n(t)\right).}
In words the softmax weighted sum of values approximates these values maximum.} computation on the other hand, one comes to the same equation\footnote{See, e.g., \url{https://en.wikipedia.org/wiki/Softmax\_function}.} which writes: \begin{equation}\label{exp-network}\begin{array}{rcl}
x_n(t) &=& \frac{e^{z_n(t)}}{\sum_n e^{z_n(t)}} = \exp\left(z_n(t) - \log(\sum_n \exp\left(z_n(t)\right)\right) \\
z_n(t) &=& \alpha_n + \sum_{n' = 0}^{N} W'_{nn'} \, x_{n'}(t-1) + \sum_{m = 0}^{N} W_{nm} \, i_m(t-1) \\
\end{array} \end{equation} with $\sum_n x_n(t) = 1$ in relation with the so-called partition function $Z(t) = \sum_n \exp\left(z_n(t)\right) > 0$. 

This kind of unit, in addition to NLN units, or LSTM units form the basic components of deep-learning architectures \cite{Bengio:2009,Deng:2014}.

The 1st line is a firmware global equation\footnote{It is worthwhile mentioning that:
\eqline{\partial_{z_{n'}(t)} o_n(t) = o_n(t) \, \left(\delta_{n=n'} - o_{n'}(t)\right) \in [0,1], \delta_{n=n'} = \left\{\begin{array}{cl} 1 & n = n' \\ 0 & \mbox{otherwise} \end{array}\right.,}
thus numerically well defined, with no singularity, the transformation being contracting, i.e., $\left|\partial_{{\bf z}} {\bf o}\right| \leq 1$, with $\max|\partial_{{\bf z}} {\bf o}|=1$.} which is a function of all units value of the same layer.

We encounter such a construction in restricted Boltzmann machine (RBM) (also using LNL network with the logistic sigmoid, but in a context of stochastic activation of the units in this case) \cite{Bengio:2009}. We mention this possibility for the completeness of the discussion, making explicit the fact that the present framework includes such equation. However, the estimation problem addressed in RBM completely differs (as being a stochastic estimation paradigm) from the deterministic estimation considered here, the key difference being the fact we want relevant results event on small data sets.

\paragraph{Other aspects of the proposed notation} 

It is also straightforward to verify that the reservoir computing equations \cite{verstraeten-etal:07} also fit with this framework, as being a particular of LNL network, since they simply correspond to a recurrent reservoir of interconnected units, plus a read-out layer.

Since there is no restriction on the architecture, depending on the choice of the kernels, it also can represent a two-layers non-linear network, or even better a multi-layers deep network. The trick is simply to choose kernels corresponding to the desired inter-layer and intra-layer connectivity.

A step further, in a given architecture, we can adjust both the number of layers and the choice between one or another computation layer. This aspect if further discussed in \cite{Drumond2017From}. We also would like to consider not only a sequence of layers, but a more general acyclic graph of layers, noticing that shortcuts can strongly improve the performance thanks to what is called residual-learning \cite{He2016Deep}. Following \cite{Fdrumond2017}, the key-point is that we want to have this structural optimization as a parameter continuous adjustment and not a meta-parameter combinatory adjustment. The proposal is thus to consider an architecture with {\em versatile layers} where the choice of the non-linearity is performed via a linear combination, obtained with sparse estimation, thus acting as a soft switch. Furthermore, adding shortcuts allows to define an adjustable acyclic graph with the output as supremum and the input as infimum. On the reverse, \cite{Fdrumond2017} points out that any acyclic graph can obviously be defined in this framework. Of course, we do not expect this method to generate the best acyclic graph and combination of modules, but to improve an existing architecture by extending usual optimization to the exploration of structural alternatives.


\clearpage\section{Comparison with related recurrent weight estimation methods} \label{backpropag}

In this section we briefly discuss how this method compares with existing methods of recurrent weight methods estimation.

The back-propagation through time (BPTT) is a gradient-based technique used, .e.g., in Elman's Networks \cite{elman:90}, where the standard back-propagation algorithm is applied to both the network recurrent layers and through time. It is based on the propagation of the error gradient, and it generally remains on two assumptions that the cost is additive with respect to training examples and that it can be written as a function of the network output (see, e.g., \cite{Nielsen2015}). With respect to this basic method, our method:
\\- does not rely on the cost gradient propagation, but the error backward propagation (or tuning), while gradients remain local to a unit.
\\- has been stated including for non additive costs (such as statistical criteria) and for both supervised criterion based on the network output error, or other unsupervised criteria.

Our formulation has been formalized, by, e.g. \cite{cun_theoretical_1988}, but without proposing a second order estimation method, considering explicitly the backward tuning of the error with a heuristic to avoid extinction and explosion. Moreover, the fact this formalism has been applied on the formulation propose in section~\ref{position} with intermediate variables makes the backward tuning proposal more efficient, than if non linearity and weights linear combination have been mixed.

Furthermore, as made explicit in \cite{doi:10.1162/089976603762552988} when comparing back-propagation with contrastive Hebbian learning, or in \cite{cun_theoretical_1988}, our backward tuning mechanism corresponds gradient back-propagation up to a change of variable. However contrary to \cite{doi:10.1162/089976603762552988} or \cite{Hochreiter:1997}, there is no need to introduce further approximation (such as, e.g, only considering diagonal terms) in order to write the backward propagation rule. This variant is well-founded, simpler to write and seems to be numerically more stable.

A step further, artificial neuron network back-propagation has been related to biological back-propagation in neurons of the mammalian central nervous system (see, e.g., \cite{Stuartetal1997}) and it is clear that the propagation of a learning or adaptive error, is more likely to be related to backward tuning of an error, than an energy or criterion gradient minimization. Regarding biological plausibility, our method only involves local distributed adjustments, as a version of back-propagation that can be computed locally using bi-directional activation recirculation \cite{HintonMcClelland1988} instead of back-propagated error derivatives is more biologically plausible, and has been improved by \cite{OReilly1996recirculation}. In its generalized form it also communicates error signals, being inspired by contrastive learning, and using the Pineda and Almeida algorithm \cite{Pineda1987}. All these methods operate on the current estimate of the derivative of the error, not the backward tuning error defined here, while related to specific cost function.

The proposed method also enjoy an interesting interpretation related to the 2nd order estimation method, as made explicit in footnotes${}^{\ref{improvingstate}}$ and ${}^{\ref{improvingkappa}}$. Thanks to the simple formulation, and either from the backward tuning of the estimation error in the case of footnote${}^{\ref{improvingstate}}$ or by direct estimation in the case of footnote ${}^{\ref{improvingkappa}}$ we obtain an estimation not only of the output desired value, but also of hidden state desired value. This corresponds to a deterministic estimation / minimization algorithmic scheme : estimation of the desired hidden state value, given the current weight values followed by the local minimization of the criterion adjusting the unit weights.

As it, even if in relation with the usual standard back-propagation method, the proposed method is a real alternative.




\clearpage\section{Using this framewrok in different contexts} \label{application}

In this section we review classical mechanism of estimation that can make use of the previous estimation mechanism. 

\subsection*{Considering a supervised learning paradigm.}

If we focus on a supervised learning paradigm, we consider learning sequences of size $T$ with desired output $\bar{\bf o}(t), 0 \le t < T$, corresponding to the input $\bar{\bf i}(t)$, in order to adjust the weights. 

This setup includes without loss of generality the possibility to use several epochs (i.e., several sequences): They are simply concatenated with a period of time with state reset at the end of each epoch, in order to guaranty to have independent state sequences.

With respect to desired output $\bar{o}_n(t)$ we can write:
\[ 
\rho_{nt}(\hat{x}_n(t)) = \frac{\kappa_{nt}}{2} (\hat{x}_n(t) - \bar{o}_n(t))^2
\]

On one hand, we choose $\kappa_{nt} > 0$ if $\bar{o}_n(t)$ is defined (output node) and $\kappa_{nt} = 0$ otherwise (hidden unit, missing data, or segmentation of the sequence in different epochs, see Fig.~\ref{epoch-concatenation}), while since $\kappa_{nt} \in [0, +\infty[$ it can also act as error gain, taking related precision into account. 

\begin{figure}[!ht]
  \includegraphics[width=0.8\textwidth]{img/epoch-concatenation}
  \caption{If the supervised learning is performed with different epoch of data, this is equivalent to a unique epoch, providing a reset segment of length $R$, the maximal recurrent range, is inserted before each new epoch. During reset segment, we set $\kappa_{nt}  = 0$.}
  \label{epoch-concatenation}
\end{figure}

One aspect of the estimation is related to robustness, i.e., being able to take into account the fact that errors and artifacts may occur in the learning set, implemented here as a M-estimator, i.e., not  a least-square function but another alternative cost function, with a smaller slope for higher values.

\subsubsection*{Considering static estimation.}

The present framework stands for dynamic estimation of a temporal sequence. It can also simply be applied to a static estimation at the final time step $T-1$ considering $\bar{o}_n(T-1)$ only the previous values $o_n(t)$ being unconstrained. In that case the value $T$ corresponds to the number of iteration to obtain the desired estimation. In a non-recurrent architecture this value is easy to derive from the architecture, it corresponds to the number of computation steps. In a recurrent architecture, the situation is more complex since computation loops have to converged, and the number of computation steps is an explicit parameter, unless the system is tuned to converge to a fixed point, while considering $T \rightarrow +\infty$ which is a rather straightforward extension of the present work.

\subsection*{Considering constrained architecture and weights values.}

It is precious to also introduce constraints on the connection weights. Typical constraints include: 
\\- sparse connectivity, which reduces the total amount of computation, and allows internal sub-assemblies to emerge, 
\\- positive or negative weight values (corresponding to excitatory or inhibitory connections).

The design choice of the kernels allows us to constraint the network connectivity. It is possible to specify partial connectivity allowing to distinguish different layers (e.g. hidden layers not connected to input and/or output).
This may be, for instance, a 2D-topography with local horizontal connections, or several layers with, e.g., either point to point, or divergent connectivity between layers.

However, if the architecture itself has to be learned, the present framework may be used in another way: Starting from a given connected network and performing a sparse estimation, may lead to a result with zero weight values for connections not present in the estimated architecture, and non zero values otherwise. This is a sparse estimation, i.e. not only minimizing the metric not only with respect to the weights values, but also with respect to the fact that some weights have either zero or non-zero values, i,e, with respect connection sets. Sparse estimation methods (see e.g. \cite{tropp:04a,tropp:04b} for a didactic introduction) can be used to this end. 

One application could be modulatory weighted connections, allowing to enhance or cancel sub-parts of the network connectivity.

In our case we may simply choose, for some meta-parameters $\nu_{nd}$: 
\[
{\cal R}({\bf W}) = \sum_{nd} \frac{\nu_{nd}}{\epsilon + |\hat{W}_{nd}|} \, W_{nd}^2
\]
where $\hat{W}_{nd}$ stands for the best a-priory or previous estimation of the weight. This leads to a reweighted least-square criterion, where small weights value minimization is reinforced, up to $0$, yielding sparse estimation.

The case where we consider excitatory or inhibitory connections (i.e., weight values that only positive or negative), or the case where the weights are bounded, is managed at the implementation level, as a hard constraint in the minimization. Very simply, if the value is beyond the bound it is reprojected on on the bound. This may lead to a sub-optimal estimation, but avoids the heavy management of Karush-Kuhn-Tucker conditions.

As an example, let us consider the adjustable leak $\gamma_{nt}$, $0 \leq \gamma_{nt} \leq 0.99 \simeq 1$ of a NLN unit. If the minimization process yields a negative value, the value is reset to zero (it means that we better have no leak). If the minimization process yields an unstable value higher than one, it is reset to, say, $0.99$ to be sure the system will not diverge.

\subsection*{Considering un-supervised regularization.}

In order to find an interesting solution, we have to constraint the hidden activity to be estimated. Interesting properties includes sparseness, orthogonality, robustness and bounds.

Sparse activity (i.e., with a maximal number of values closed or equal to zero), which is known to correspond to unit assemblies tuned to a given class of input statistics, can be specified as a reweighted least-square criterion again, for some meta-parameters $\kappa_{nd}$:
\[
\rho_{nt}(x_{nt}) = \frac{\kappa_{nd}}{\epsilon + |\hat{x}_{nt}|} \, x_{nt}^2
\]
where $\hat{x}_{nd}$ stands for the previous estimation, with an initial value equal to $\kappa_{nd}$.

Orthogonality of hidden unit activities, in order to avoid redundancy and maximize the dynamic space dimension in the recurrent network, can also be specified, the same way as :
\[
\rho_{nt}(x_{nt}) = \kappa_{nd} \, \sum_{n' \neq n} (\sum_t x_{nt} \, \hat{x}_{n't})^2
\]
again as as, now not local but global, reweighted least-square criterion, now minimizing the dot products between unit activities, thus minimal when orthogonal. 

Another aspect concerns the fact we may have to control the activity bound, e.g., a weak constraint of the form $x_{nt} \preceq b$. Following the same heuristic, we may introduce a cost of the form:
\[
\rho_{nt}(x_{nt}) = \kappa_{nd} \, e^{k\,(x_{nt} - b)}
\]
with $k > 0$ in order to have a fast increasing function as soon as the bound is violated.

\clearpage\section{Closed forms solution for neural network tasks} \label{closedforms}

Let us illustrate how the type of used units has a strong influence on the difficulty of the task. Here we consider deterministic tasks only. The remark is that tasks considered as quite complex \cite{Hochreiter:1997,Gers:2003,martens_learning_2016} for certain architectures are trivial for others. In particular, the use of AIF neurons simplifies certain problems, e.g. requiring long short-term memory. We illustrate this point here considering deterministic sequence generation and long-term non-linear transform, and provide explicit simple solutions for those problems.

\subsubsection*{Generating long term sequential signals}

The lever is that it is straightforward to generate a delayed step signal (i.e., equal to 0 before $t=\tau$, and 1 after) using AIF units, e.g.:
\eqline{s_\tau(t) = \frac{1}{2} \, (1 - \Upsilon(s_\tau(t-1))) \, s_\tau(t-1) + h_\tau}\\
with 
\eqline{h_\tau \deq \frac{1}{4\, \left(1 - 2^{\frac{1}{2}-\tau}\right)} \in [h_\infty = 1/4, h_1 \simeq 0.85],}\\  
for which we easily obtain\footnote{{\bf Delayed step signal.} Starting with $s_\tau(0) = 0$ this first order recurrent equation yields: 
\eqline{s_\tau(t) = 2 \, h \, \left(1 - 2^{-t}\right) \in [0, 2\, h],} 
which is an bounded increasing negative exponential profile, for which the parameter $h$ has been chosen to maintain $s_\tau(t) < 1/2, t < \tau$, and reach $s_\tau(t) > 1/2$, for $t \geq \tau$.
\hr} $\Upsilon(s_\tau(t)) = \delta_{t\geq\tau}$. 

The numerical limit of this method is the fact that for huge value of $\tau$ the parameter precision must be of order $O\left(2^{-\tau}\right)$. To avoid this constraint, either an architecture with several units building a delay line, or with a ramp unit and adaptive thresholds (see next section) can be considered.

From this basic element we can generate a delayed clock signal\footnote{{\bf Delayed clock signal.} Modifying the delayed step signal, and adding a memory carousel unit in order to reset the signal after the step and keep it reseted, we obtain: 
\eqline{\begin{array}{rcl}
 c_\tau(t) &=& \frac{1}{2} \, (1 - \Upsilon(c_\tau(t-1))) \, c_\tau(t-1) + h_\tau \, (1 - \Upsilon(d_\tau(t-1))) \\
 d_\tau(t) &=& d_\tau(t-1) + \Upsilon(c_\tau(t-1)), \\
\end{array}}
with $\Upsilon(c_\tau(t)) = d_\tau(t) = 0, t < \tau$, until $c_\tau(\tau) > 1/2$, As a consequence $d_\tau(\tau + 1) = 1$, thus $c_\tau(\tau + 1) = 0$, which is a stable fixed point, values remaining constant beyond. Finally we obtain $\Upsilon(c_\tau(t)) = \delta_{t=\tau}$ in this case.
\hr} or another long-term mechanism, such a as flip-flop\footnote{{\bf Defining a flip-flop latch.} Let us defined a SR-latch (i.e., a flip-flop) with:
\eqline{\begin{array}{rcl}
z(t) &=& \Upsilon(z(t-1)) + \Upsilon(i_1(t)) - \Upsilon(i_0(t))\\
\end{array}}
yielding the following behavior:
\\- {\em R-state}: If $i_0(t) < 1/2$ and $i_1(t) < 1/2$ (no-input) and $z(t-1) < 1/2$, then $z(t) = 0 < 1/2$, the reset state is maintained.
\\- {\em S-state}: If $i_0(t) < 1/2$ and $i_1(t) < 1/2$ (no-input) and $z(t-1) > 1/2$, then $z(t) = 1 > 1/2$, the set state is maintained.
\\- {\em R-S transition}: If $i_0(t) < 1/2$ and $i_1(t) > 1/2$  and $z(t-1) < 1/2$, then $z(t) = 1 > 1/2$, flipping to a set state; if it was already in the set state, we still have $z(t) = 2 > 1/2$.
\\- {\em S-R transition}:If $i_0(t) > 1/2$ and $i_1(t) < 1/2$  and $z(t-1) > 1/2$, then $z(t) = 0 < 1/2$, flipping to a reset state; f it was already in a reset state, we still have $z(t) = - 1 < 1/2$.
\\- {\em no instability}: $i_0(t) > 1/2$ and $i_1(t) > 1/2$ contrary to a standard digital RS-latch we simply have $z(t) = \Upsilon(z(t-1))$ providing it was in set of reset state, without any meta-stability.
\hr}, which is a fundamental building blocks of any digital transform, in conjunction with logic gates such as a xor gate\footnote{{\bf Defining the xor function.} It is straightforward to notice that: 
\eqline{\begin{array}{rcl}
x_\bullet(t) &=& \Upsilon(x_a(t - 1)) + \Upsilon(x_b(t - 1)) + -2 \, \Upsilon(x_o(t)) \\
x_o(t) &=& \Upsilon(x_a(t - 1)) + \Upsilon(x_b(t - 1)) - 1 \\
\end{array}}
verifies 
\eqline{\begin{array}{rcl}
\Upsilon(x_o(t))  &=& \Upsilon(x_a(t - 1)) \mbox{ and } \Upsilon(x_b(t - 1)) \\
\Upsilon(x_\bullet(t))  &=& \Upsilon(x_a(t - 1)) \mbox{ xor } \Upsilon(x_b(t - 1)) \\
\end{array},}
while other logic gates are easy to build in a similar manner.\\ A step further the expression 
\eqline{x_\dagger(t) = 1/2 - 2 \, (\Upsilon(x_a(t - 1)) - 1/2) \, (\Upsilon(x_b(t - 1)) - 1/2)}
now considering a multiplication unit, directly calculates the xor function, but does no correspond to some AIF unit.
\hr}. 

If we consider a mollification instead of a step function (i.e., replacing $\Upsilon$ with$\Upsilon_\epsilon$ in the previous equation), we obtain the behavior for sufficiently large slopes. More precisely\footnote{This is obtained, e.g., by the following piece of maple code: {\tt upsilon := (u) -> 1/(1+exp(-4*(u - 1/2)/epsilon)):\\
c\_n := c -> (1 - upsilon(c)) * c / 2 + h:\\
bounds := [solve(c\_n(1/2) = 1/2, h), solve(c\_n(0) = 1/2, h)];\\}\hr}, for instance, we numerically observed the same qualitative behavior in the delayed step signal case, with $h \in [h_\infty = 0.376, h_1 = 0.5]$, while $h$ is not given in closed form in this case.

Further on this track, it is clear that we can compile any sequential circuit in such networks, which is far from being new. The add-on here is about that the fact we provide explicit solutions, using AIF neurons, with a lower complexity in terms of network nodes than using LSTM units. Let us see two paradigms where this enlighten the problem complexity.

\subsubsection*{Long term non-linear transform}

In many experiments, a variant of a sequence of the form:
\\\centerline{\begin{tabular}{lcccccccccccc}
time :  & 0   & 1   &     &          &     & T\\
input:  & $a$ & $b$ & $*$ & $\cdots$ & $*$ & $*$\\
output: & $*$ & $*$ & $*$ & $\cdots$ & $*$ & $a \, b$ \\
\end{tabular}}\\
where $a$ and $b$ are variable input, $*$ are random distractors and $a\,b$ the desired delayed output (here a product, but it could be another calculation). Such setup combines several non-trivial aspects, long short term memory, distractor robustness, and operation which may not explicitly hardwired in the network, presently a product. The LSTM approach was shown to be particularly efficient for such computation, because of the notion of ``memory carousel''. In fact, the explicit implementation of such a mechanism on the given example is trivial\footnote{{\bf An example of long term computation.} One solution writes: 
\eqline{\begin{array}{rcl}
o_0(t) &=& (1 - \Upsilon(c_T(t)) \, i(t) + (\Upsilon(c_T(t)) - 1) \, x_a(t) \, x_b(t) + \\
x_a(t) &=& (1 - \Upsilon(c_0(t)) \, x_a(t-1) + (\Upsilon(c_0(t)) - 1) \, i(t) \\
x_b(t) &=& (1 - \Upsilon(c_1(t)) \, x_b(t-1) + (\Upsilon(c_1(t)) - 1) \, i(t) \\
\end{array}} 
while $c_\tau(t) = \delta_{t = \tau}$ are clock signals, as defined previously, and it is easy to verify that $x_a(t)$ ``opens'' the memory at time $t=0$, and stores the previous value otherwise, with a similar behavior for $x_b(t)$, while $o_0(t)$ simply mirror the input until $t=T$, where the expected result is output. Obviously, these are no more AIF units but introduce multiplications between state values}.

What do we learn from this very simple development? While authors have already made explicit the fact that such computations rely on ``gate unit'' and ``memory unit'', it seems that ``delayed unit'' (i.e. learning a time delay) are also basic components. It is also an example of how deterministic computations might become simple, if we introduce a-priory information on the computation, via dedicated units.

\subsubsection*{Deterministic sequence generation}

What is the complexity of the task of generating a deterministic time sequence $\bar{o}_n(t), n \in \{0, N_0\{, t \in \{0, T\{$, with a recurrent network of $N \ge N_0$ units of range $R$? This could be an unpredictable sequence, without any algorithm to generate it, unless copying all sample (i.e., with a maximal Kolmogorov complexity).

On one hand, $O(N_0)$ independent linear recurrent units of range $R=T$, solves the problem of generating an exact sequence of $N_0 \, T$ samples, in closed form\footnote{{\bf Long range sequence generation.} Let us consider units of the form:
\eqline{x_n(t) = \sum_{d = 1}^{d = T-1} W_{nd} \, x_n(t-d) + W_{n0},}
thus with $N_0 \, T$ weights. Since $x_n(t) = 0, t < 0$, providing $\bar{o}_n(1) \neq 0$, we immediately obtain $W_{n0} = \bar{o}_n(1)$ and for $d > 0$: 
\eqline{W_{nk} = (\bar{o}_n(k+1) - W_{n0} - \sum_{d = 1}^{d = k-1} W_{nd} \, \bar{o}_n(t-d)) / \bar{o}_n(1),}
providing that $\bar{o}_n(1) \neq 0$, thus a closed-form solution. If $\bar{o}_n(1) = 0$ we simply have to generate the sequence, say, $\bar{o}'_n(t) = \bar{o}_n(t) + 1$ and add a second unit of the form $x_n(t) = x'_n(t) - 1$, using now an additional node.
\hr}. This solution requires a very large recurrent range, and the numerical precision is limited by the fact that errors accumulate along the recurrent calculation.

On the other hand, feed-forward units of range $R=1$ solve explicitly the problem using $T$ clock units and $N_0$ readout units, with $O(N_0\,T)$ weights. This requires no more than $N_0+T$ units considering binary information\footnote{{\bf Long sequence generation with delay lines.} Let us consider $N_0$ readout units and $T$ clock units of the form: 
\eqline{\begin{array}{rcll}
x_{n_0}(t) &=& \sum_{n = 0}^{T-1} (\bar{o}_{n_0}(n) - \bar{o}_{n_0}(n+1)) \, \Upsilon(x_{N_0+n}(t)) & 0 \leq n_0 < N_0, \mbox{ writing }  \bar{o}_{n_0}(T) \deq 0\\
x_{N_0}(t) &=& \frac{1}{2} \, (1 - \Upsilon(x_{N_0}(t-1))) \, x_{N_0}(t-1) + h_1 \\
x_{N_0+n}(t) &=& x_{N_0+n-1}(t-1) & 0 < n < T \\
\end{array}}
thus providing $T$ delayed step signals such that $\Upsilon(x_{N_0+n}(t)) = \delta_{t>n}$, allowing us to generate the desired sequence combining these signals. If we now consider mollification of he threshold function, the previous system of equation is going to generate a temporal partition of unity. Since ${\bf x}_{N_0+n}$ are simple shifts of ${\bf x}_{N_0}$, the clock units obviously span the output signal space and output units can easily linearly adjust there related combination to obtain the desired values.
\hr}, and no more that $N_0+1$ units if the numerical precision is sufficient and unit threshold adjustable\footnote{{\bf Long sequence generation with a ramp unit.} If we can consider units of the form: 
\eqline{\begin{array}{rcrl}
x_{n_0}(t) &=& \sum_{n = 0}^{T-1} (\bar{o}_{n_0}(n) - \bar{o}_{n_0}(n+1)) \, \Upsilon(x_{N_0+n}(t) - \theta_n) \\
x_{N_0}(t) &=&  x_{N_0}(t-1) + 1\\
\end{array}}
with the ramp unit $x_{N_0}(t)$ precision being of order $O(1/T)$, while we now can introduce adaptive thresholds $\theta_n = n$, it is obvious to verify that we solve the problem with two units.
\hr}.
A step further, considering less than $N_\bullet \deq \sqrt{N_0\,T/R}$ linear or NLN units of range $R$, we can not generate a solution in the general case\footnote{{\bf Long sequence generation with fully connected network.} Considering the linear network system:
\eqline{x_n(t) = \sum_{m = 1}^{m  = N} W_{nmr} \sum_{r= 1}^{r=R} \, x_m(t-r) + W_{n0},}
with $0 \leq n_0 < N_0$ output units and $N_0 \leq n < N$ hidden units, using vectorial notations, with the shift operator ${\cal S}$ defined as ${\cal S} {\bf x}(t-1) = {\bf x}(t)$, we obtain:
\eqline{{\cal S} \, \begin{array}{r}{}_{N_0\,T}\updownarrow\\ {}_{(N-N_0)\,T}\updownarrow \end{array}
\left(\begin{array}{c} \bar{\bf o} \\ \bar{\bf x} \end{array}\right) = {\bf W} \, \left(\begin{array}{c} \bar{\bf o} \\ \bar{\bf x} \end{array}\right) + W_{0}}
where $\bar{\bf o}$ are the desired output. It is a bi-linear system of $N\,T$ equations in $N^2\,R + N$ independent unknowns, i.e., the weights, while the $(N-N_0)\,T$ hidden values are entirely specified as soon as the weights are given. In terms of number of degree of freedom we can not have $N^2\,R + N < N_0 \,T$ for this algebraic system of equation to have a solution in the general case.}.

The generation of periodic signal of period $T$, is a very similar problem, as studied in \cite{rostro-gonzalez-cessac-etal:10}, for $N=N_0$. In a nutshell, we simply must add equations such that ${\bf x}(T) = {\bf x}(0)$ to guaranty the periodicity.

From this discussion, we see that the complexity of signal generation problem highly depends on the kind of ``allowed units'' and reduces to a trivial problem as soon as suitable operation are allowed. Furthermore, there exist a $R=1$ network of at most $N_0 + T$ units that exactly solves the problem, without requiring huge precision, while a linear network, a NLN network or a AIF network can generate such a sequence in the general case, with either a closed form solution, or solving a linear system of equation.

\clearpage\section{Stochastic adjustment of the network weights}\label{stochastic}

Let us consider the problem of optimizing the probabilistic distribution $\tilde{p}(\tilde{\bf x} = {\bf x})$ of a network output, as a function of the desired distribution $\bar{p}(\bar{\bf o} = {\bf o})$ of a root network. Since we are in a multi-dimensional and dynamic framework, with continuous values, it is  intractable  to consider as it the distribution, but only a parametric model of it, and adjust the parameters of this model. 

\subsubsection*{An illustrative example}

Let us, for instance, consider that it is important that the network output mean $\tilde{\Omega}_{n,\bullet}$ and auto-correlation in a $\tau = \{0, \Delta\}$ time window $\tilde{\Omega}_{n,\tau}$, correspond to some desired values:
\eqline{\begin{array}{rclrcl}
   \bar{\Omega}_{n,\bullet}  &\deq& \frac{1}{T} \, \sum_{t = 0}^{T - 1} \omega_{n,\bullet}(t), & \omega_{n,\bullet}(t) &\deq& \bar{o}_n(t) \\
   \bar{\Omega}_{n,\tau}  &\deq& \frac{1}{T - \tau} \, \sum_{t = 0}^{T - \tau - 1} \omega_{n,\tau}(t) & \omega_{n,\tau}(t) &\deq& \bar{o}_n(t) \, \bar{o}_n(t - \tau), \\
\end{array}}
the normalized temporal auto-correlation being: 
\eqline{C_{n,\tau} = (\bar{\Omega}_{n,\tau} - \bar{\Omega}_{n,\bullet}^2) / (\bar{\Omega}_{0,\tau} - \bar{\Omega}_{n,\bullet}^2).}
We thus do not constraint the output desired values directly but only some momenta expectation.

Beyond this example, we thus consider observable $\omega_{k}(x_n(t) \cdots x_n(t-\tau_k))$ of a given rank $\tau_k$ and their expectation on the desired distribution $\Omega_{k}$. We could also have considered higher order momenta, e.g., consider mean, standard-deviation, skewness and kurtosis, or spatial correlations, and so on.

\subsubsection*{Considering a general model}

Adapting the development given in \cite{vasquez:inria-00574954} for binary distribution, we propose to minimize the KL-divergence, considering maximal entropy Gibbs distributions. We are going to propose to adjust the network weights in order to minimize an approximation of the KL-divergence between the desired and simulated distribution.

If we look for a distribution of probability with maximal entropy and which observable $\omega_{k}$ correspond to some expectation values ${\Omega}_{k}$, we obtain:
\eqline{p({\bf x}) = \frac{\exp\left(\sum_k \lambda_k \, {\omega}_{k}({\bf x}) \right)}{Z_p({\bf \lambda})},}
where the denominator guaranties $\int_{{\bf x}} p({\bf x}) = 1$ and is called the partition function\footnote{{\bf Maximal entropy distribution.} Given expectation  ${\Omega}_{k}$  of observable $\omega_{k}(t)$ we state that we look for a probability distribution of maximal entropy which corresponds to the observable expectation. This writes, with Lagrangian multipliers $\lambda_k$:
\eqline{\min_{\bf \lambda} 
\underbrace{\int_{\bf x} p({\bf x}) \log(p({\bf x}))}_{entropy} +
\underbrace{\lambda_{0} \left(\int_{\bf x} p({\bf x}) -1 \right)}_{normalization} -
\underbrace{\sum_k \lambda_{k} \left(\int_{\bf x} p({\bf x}) \, \omega_{k} - {\Omega}_{k}\right)}_{observations}}
and the functional derivative of this criterion yields:
\eqline{p({\bf x}) = \exp\left(\sum_k \lambda_k \, {\omega}_{k}({\bf x})\right) / Z_p({\bf \lambda}),}
as easily obtained from the normal equation derivation, see e.g.:
\\ \centerline{\href{https://en.wikipedia.org/wiki/Maximum\_entropy\_probability\_distribution\#Proof}{https://en.wikipedia.org/wiki/Maximum\_entropy\_probability\_distribution\#Proof}.}\hr}, topological pressure or free energy. The quantity $Z_p({\bf \lambda})$ has no closed form beyond simple cases, and can be numerically estimated as:
\eqline{Z_p({\bf \lambda}) = \int_{\bf x} \exp\left(\sum_k \lambda_k \, {\omega}_{k}({\bf x})\right) \simeq \frac{1}{T - \tau} \, \sum_{t = 0}^{T- \tau} \exp\left(\sum_k \lambda_k \, {\omega}_{k}(t)\right)}, under the ergodic assumption, $\tau$ being chosen for all observable ${\omega}_{k}(t)$ to be defined.

\subsubsection*{Fitting a Gibbs distribution}

A step further, it appears that minimizing the KL-divergence between the observed distribution $\bar{p}(\bar{\bf o})$ and the Gibbs model corresponds to adjust the parameters $\bar{\bf \lambda}$ in order the predicted observable expectation $\Omega_k({\bf \lambda})$ to get as closed as possible to the desired observable expectation $\bar{\Omega}_{k}$, which is a standard estimation problem (in a nutshell, the trick is to minimize the criterion gradient, not the criterion itself\footnote{{\bf Fitting the Gibbs parameters distribution}. For the sake of completeness, let us detail how such estimation can be performed. If we consider the KL-divergence between the observed distribution $\bar{p}(\bar{\bf o})$ and the model approximate distribution $q({\bf x}))$, we easily derive: 
\eqline{\begin{array}{rcl}
d_{KL}(\bar{p}(\bar{\bf o})\|q({\bf x})) 
&=& \int \bar{p}(\bar{\bf o}) \, \log\left(\frac{\bar{p}(\bar{\bf o})}{q({\bf x})}\right) \\
&=& \int \bar{p}(\bar{\bf o}) \, \log\left(\bar{p}(\bar{\bf o})\right) - \int \bar{p}(\bar{\bf o}) \, \log\left(q({\bf x})\right) \\
&=& -h_{\bar{\bf o}} - \int \bar{p}(\bar{\bf o}) \log\left(q({\bf x})\right) \\
&=& -h_{\bar{\bf o}} - \int \bar{p}(\bar{\bf o}) \left(\sum_k \lambda_k \, \omega_{k} - \log(Z({\bf \lambda})) \right) \\
&=& -h_{\bar{\bf o}} - \sum_k \lambda_k \, \bar{\Omega}_{k} + 1 \, \log(Z_q({\bf \lambda})) \\
\end{array}}
combining the previous equations, and since the term $h_{\bar{\bf o}} \deq -\int \bar{p}(\bar{\bf o}) \log(\bar{p}(\bar{\bf o}))$ is the observed entropy and is constant with respect to the parameter to estimate, we are left with the following criterion, which in fact corresponds to cross-entropy maximization $\min_{\bf \lambda} {\cal J}$, with: 
\eqline{\begin{array}{rcl}
 {\cal J} &=& \log\left(Z_q({\bf \lambda})\right) - \sum_k \lambda_k \, \bar{\Omega}_{k} \\
 \partial_{\lambda_k} {\cal J} &=& \Omega_k({\bf \lambda}) - \bar{\Omega}_{k} \\
 \partial_{\lambda_k\, \lambda_l} {\cal J} &=& \Omega_{kl}({\bf \lambda}) \\
\end{array}}
writing $\omega_{kl}(t) = \omega_k(t) \, \omega_l(t)$, and $\Omega_{kl} = {\mathbb E}[\omega_{kl}]$. This computation comes from the fact that:
\eqline{\begin{array}{rcl}
Z_q({\bf \lambda}) 
&=& \int_{{\bf x}} \exp\left(\sum_k \lambda_k \, {\omega}_{k}({\bf x}) \right) \\
\partial_{\lambda_k} Z_q({\bf \lambda}) 
&=& \int_{{\bf x}} \exp\left(\sum_k \lambda_k \, {\omega}_{k}({\bf x}) \right) \, {\omega}_{k}({\bf x}) \\
&=& \int_{{\bf x}} Z_q({\bf \lambda}) \, q({\bf x}) \, {\omega}_{k}({\bf x}) \\
&=& Z_q({\bf \lambda}) \, {\Omega}_{k}({\bf \lambda}),
\end{array}}
and it is easy to approximate:
\eqline{\begin{array}{lrcl}  
\Omega_{l}({\bf \lambda}) 
&\deq& \int_{\bf x} \frac{\exp\left(\sum_k \lambda_k \, \omega_{k}\right)}{Z_q({\bf \lambda})} \omega_l(t)\\
&\simeq& \frac{1}{T - \tau_l} \, \sum_t \omega_l(t) \\
\end{array}}
under the ergodic assumption.

As a consequence, despite the caveat that $Z_q({\bf \lambda})$ calculation is usually not tractable, this allows us to implement some paradigm that tends to minimize the criterion gradient (since at a criterion minimum, the gradient vanishes):
\eqline{\bar{\lambda} = \mbox{arg min}_\lambda |\Omega_{l}({\bf \lambda}) - \bar{\Omega}_{k} |.}
One example of algorithm writes:
\\ \hphantom{2mm} {\em Input} : The desired observable values $\bar{\Omega}_{k}$ and the distribution samples $\bar{\bf o}$.
\\ \hphantom{2mm} {\em Output} : The estimated $\bar{\lambda}_k$.
\\ \hphantom{4mm} - Starts with ${\bf \lambda}_0 = 0$ and a regularization parameter $\upsilon = 1$.
\\ \hphantom{4mm} - At a given iteration $i$
\\ \hphantom{4mm} -- Computes $\Omega_k({\bf \lambda})$ and $\Omega_{kl}({\bf \lambda})$ for a given value of ${\bf \lambda}$ from a random draw $\pi(t)$.
\\ \hphantom{4mm} -- In order to obtain ${\bf \lambda}_{i} = d{\bf \lambda} + {\bf \lambda}_{i-1}$ solve the regularized linear problem: 
\eqline{d{\bf \lambda} = \mbox{arg min}_{d{\bf \lambda}} |d{\bf \lambda}|, \;\;\; \upsilon \, \partial {\cal J} + (1 - \upsilon) \, \partial^2 {\cal J} \, {\bf \lambda}_{i-1} = \partial^2 {\cal J} \, d{\bf \lambda}}
calculating the SVD of $\partial^2 {\cal J}$ in order to consider its pseudo-inverse.
\\ \hphantom{4mm} -- If $\|\partial {\cal J}\|$ does not decreases reduce $\upsilon$ and repeat until $\upsilon$ vanishes.
\hr}). As a consequence, given a desired output $\bar{\bf o}$ and a choice of observable $\omega_k$ we can estimate the maximal entropy parameters $\bar{\bf \lambda}$.

\subsubsection*{Statistical weight adjustment from the parametric model}

Given set of desired observable values $\bar{\Omega}_k$, with the corresponding Gibbs model $\hat{p}(\bar{\bf o})$ parameterized by $\bar{\bf \lambda}$ and adjusted on the reference samples $\bar{\bf o}$, we now can state the problem of adjusting the network weights. We consider the KL-divergence between the observed distribution $\bar{p}(\bar{\bf o})$, approximated by the related Gibbs model, and the network simulation $\tilde{p}_{\bf W}(\tilde{\bf x})$, parameterized by the network weights ${\bf W}$. The network is viewed here as a parametric model of the observed distribution. 

Since the network simulation is brought to the desired reference samples distribution, modeled as a Gibbs distribution, we are going to assume that the network simulation can itself be represented by a Gibbs distribution:
\eqline{\tilde{p}(\tilde{\bf x}) \simeq \exp\left(\sum_k \tilde{\lambda}_k \, {\omega}_{k}({\bf x})\right) / Z_{\tilde{p}}(\tilde{\bf \lambda}),}
yielding, using similar algebra as before:
\eqline{\begin{array}{rcl} d_{KL}(\bar{p}(\bar{\bf o})\|\tilde{p}(\tilde{\bf x})) 
 &=& \int p(\bar{\bf o}) \log\left(\frac{\bar{p}(\bar{\bf o})}{\tilde{p}(\tilde{\bf x})}\right) \\
 &\simeq& \int p(\bar{\bf o}) \log\left(\frac{\hat{p}(\bar{\bf o})}{\tilde{p}(\tilde{\bf x})}\right) \\ &=& \sum_k (\bar{\lambda}_k - \tilde{\lambda}_k) \, \bar{\Omega}_{k}
          + \log(Z_{\tilde{p}}(\tilde{\bf \lambda})/Z_{\bar{p}}(\bar{\bf \lambda})) \\
\end{array}}
with the goal to adjust the weights in order the related $\tilde{\bf \lambda}$ to minimize this divergence. As before, we can replace the KL-divergence minimization by the minimization of the gradient magnitude. This design choice is valid because the topological pressure is convex with respect to ${\bf \lambda}$, so that the criterion is convex \cite{vasquez:inria-00574954}. As a consequence, the criterion is minimal when the gradient magnitude vanishes, i.e. is minimal too, while the criterion decreases with the gradient magnitude, thanks to being a convex criterion.

The gradient writes $\partial_{\tilde{\lambda}_k} d_{KL}(\bar{p}(\bar{\bf o})\|\tilde{p}(\tilde{\bf x})) = \tilde{\Omega}_{k}(\tilde{\bf \lambda}_{\bf W}) - \bar{\Omega}_{k}$, and we propose to consider the following weighted ${\cal L}^1$ norm:
\begin{equation} \label{kl-criterion} \rho({\bf x}) \deq \sum_k |\bar{\lambda}_k| \, \left|\bar{\Omega}_{k} - \frac{1}{T-\tau_k} \sum_t \omega_k(t)\right|. \end{equation}
The reason of this second design choice is that it has the same order of magnitude as the $d_{KL}(\bar{p}(\bar{\bf o})\|\tilde{p}(\tilde{\bf x}))$ with respect to the observable, i.e.:
\eqline{|\partial_{\bar{\Omega}_k} d_{KL}(\bar{p}(\bar{\bf o})\|\tilde{p}(\tilde{\bf x}))| = |\partial_{\bar{\Omega}_k} \rho({\bf x})| = |\bar{\lambda}_k|,}
so that we expect the numerical condition of the original criterion and the related gradient magnitude to be similar. At the experimental level we have observed that such ${\cal L}^1$ criterion seems more efficient than the corresponding ${\cal L}^2$ criterion.

\subsubsection*{Boostrapping the network estimation}

Given the previous criterion, we consider a network simulation $x_n(t)$, with the goal to modify the related network weights in order the network output observable to be $\frac{1}{T-\tau_k} \sum_t \omega_k(t)$ to be as closed as possible to the desired observable $\bar{\Omega}_k$. To this end, we define desired values $\hat{x}_n(t)$, as follows
\eqline{\min_{\hat{x}_n(t)} \frac{1}{2} \, \sum_{nt} \|\hat{x}_n(t) - x_n(t)\|^2 + \sum_k \lambda_k \left[\bar{\Omega}_{k} - \frac{1}{T-\tau_k} \sum_t \left. \omega_k(t)\right|_{\hat{x}} \right]}
in words: The desired values are the closest values with respect to the present simulation that match the desired observable, while the normal equations yield the recurrent equation:
\eqline{\begin{array}{rcl}
  \hat{x}_n(t) &\leftarrow& x_n(t) + \sum_k \frac{1}{T-\tau_k} \, \lambda_k \, \sum_t \partial_{x_n(t)} \omega_k(t) \\
  \lambda_k &\simeq& \left[ \frac{1}{T-\tau_k} \, \frac{1}{T-\tau_{k'}} \, \sum_{nt} \partial_{x_n(t)} \omega_{k'}(t) \, \partial_{x_n(t)} \omega_{k}(t)^T \right]^{\dagger} \, \left[\bar{\Omega}_{k'} - \frac{1}{T-\tau_k'} \sum_t \omega_{k'}(t)\right] \\
\end{array}}
which is a standard non-linear constrained least-square 2nd order recurrent scheme (see e.g., \cite{vieville:inria-00074888}). In other words, we iteratively compute the projection $\hat{x}_n(t)$ of the present simulation $x_n(t)$ onto the manifold defined by $\bar{\Omega}_{k} = \frac{1}{T-\tau_k} \sum_t \omega_k(t)$, i.e., such that the desired values observables equal the desired observable values. Here, $\partial_{\bf x} \omega_{k}(t)$ corresponds to the local norma of the manifold, and since observables are mainly monomials of degree less than $D$, we can write for some $u_{nt} \in \{0, 1\}, d_{nt} \in \{0, D\}$ defining the monomial:
\eqline{\omega_{k}(t) = \prod_{nt} u_{nt} \, x_{n}(t)^{d_{nt}}, \mbox{ yielding } \partial_{x_{n't'}} \omega_k(t) = d_{n't'} \, \frac{\omega_{k}(t)}{x_{n't'}}}
which is well-defined, including for $x_{n't'} = 0$.

Such desired values may be used to provide local solutions to the estimation problem.



\clearpage{\scriptsize \bibliographystyle{plain} \bibliography{../bib/vthierry,../bib/from-keops,../bib/from-sophia}}
\tableofcontents
\iffalse
\fi

\end{document}

